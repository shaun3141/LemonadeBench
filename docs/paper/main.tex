\documentclass[11pt]{article}

% Packages
\usepackage[utf8]{inputenc}
\usepackage[T1]{fontenc}
\usepackage{amsmath,amssymb,amsfonts}
\usepackage{graphicx}
\usepackage{booktabs}
\usepackage{hyperref}
\usepackage{xcolor}
\usepackage{natbib}
\usepackage{algorithm}
\usepackage{algorithmic}
\usepackage{subcaption}
\usepackage{microtype}
\usepackage{enumitem}

% Page setup
\usepackage[margin=1in]{geometry}

% Title and authors
\title{LemonadeBench: A Dynamic Benchmark for Evaluating AI Agent Decision-Making in Simulated Business Environments}

\author{
  Shaun Van Weelden \\
  % Institution \\
  \texttt{shaun.t.vanweelden@gmail.com}
}

\date{\today}

\begin{document}

\maketitle

\begin{abstract}
Existing benchmarks for evaluating Large Language Model (LLM) agents---such as SWE-bench for code, WebArena for web navigation, and AgentBench for general reasoning---predominantly test single-session task completion or static decision-making. This gap is critical: real-world deployment contexts---from supply chain optimization to financial trading---require agents to maintain coherent strategies across extended time horizons where early mistakes compound into late-game constraints. We introduce \textbf{LemonadeBench}, a novel benchmark for evaluating LLM agent capabilities in \emph{multi-day sequential business decision-making} under uncertainty. Unlike existing economic benchmarks that probe isolated choices or simulate human preferences, LemonadeBench requires agents to operate a lemonade stand business over a 14-day season, making daily decisions about pricing, inventory procurement, location selection, and capital allocation---all while managing perishable goods, stochastic weather patterns, and a customer reputation system with delayed feedback dynamics.

Our benchmark features several properties absent from current agent evaluations: (1) \textbf{interpretable actions} requiring no domain expertise---decisions like ``buy 10 lemons'' or ``raise price to \$0.75'' can be evaluated by non-experts, unlike abstract RL actions or complex code edits; (2) \textbf{multi-factor optimization} across six interacting systems (weather, pricing, demand, perishables, locations, upgrades); (3) \textbf{compounding decisions} where day-1 choices constrain day-14 possibilities through inventory, cash flow, and reputation; and (4) \textbf{deterministic seeding} for reproducible scientific evaluation. The environment runs in milliseconds per step, enabling large-scale model comparison.

Beyond standard model comparison, we make a novel contribution by systematically investigating how \textbf{goal-framing prompts} shape agent economic behavior---testing whether motivational language (aggressive, conservative, competitive) induces measurable behavioral shifts analogous to those observed in human behavioral economics. We also examine \textbf{agent architectures} (comparing simple reactive loops against explicit planning and reflection phases) and \textbf{cognitive scaffolding} (whether calculator tools, math encouragement prompts, or code interpreters improve reasoning). Across 960 episodes with six frontier models, we find that [key finding about framing], [key finding about architecture], and [key finding about scaffolding]. Our findings reveal systematic behavioral patterns across models and provide actionable insights for practitioners deploying LLM agents in economic decision-making contexts, while raising important questions about the alignment between stated objectives and emergent agent behavior. LemonadeBench provides an intuitive, fast, and scientifically rigorous testbed for developing agents capable of sustained economic reasoning.
\end{abstract>

% Main sections
\section{Introduction}
\label{sec:introduction}

Large Language Models (LLMs) are rapidly transitioning from text generation systems to autonomous agents deployed in complex, high-stakes environments \citep{liu2023agentbench}. Evaluating these agents requires moving beyond question-answering accuracy to assess their ability to reason about consequences, plan over multiple steps, and adapt to dynamic conditions---capabilities tested by emerging benchmarks for code generation (SWE-bench), web navigation (WebArena), and general tool use (AgentBench).

However, existing benchmarks share a common limitation: they predominantly evaluate \emph{single-session task completion}. An agent either fixes a GitHub issue or it doesn't; it either navigates to the correct webpage or fails. While valuable, these evaluations miss a crucial dimension of intelligent behavior: \textbf{sustained decision-making over extended time horizons} where early choices constrain future possibilities through resource accumulation, reputation effects, and opportunity costs.

Consider how humans run businesses: today's inventory purchase affects tomorrow's ability to serve customers; this week's pricing strategy shapes next month's brand perception; a capital investment now enables efficiency gains later. These temporal dependencies create rich optimization landscapes that require balancing exploration and exploitation, managing risk under uncertainty, and maintaining coherent strategies across many decision points.

From a decision-theoretic perspective, such environments can be formalized as Partially Observable Markov Decision Processes (POMDPs) \citep{kaelbling1998planning} where optimal behavior requires reasoning about latent state (future weather), maintaining beliefs over long horizons, and solving the credit assignment problem---determining which past decisions led to current outcomes \citep{sutton2018reinforcement}. Behavioral economics further complicates the picture: human decision-makers exhibit systematic biases including loss aversion \citep{kahneman1979prospect}, herding behavior in financial markets, temporal discounting, and risk-seeking behavior under competitive pressure. A natural question arises: \emph{Do LLM agents exhibit similar biases? Can prompt engineering modulate these behaviors?} More fundamentally, if we frame an agent's objective as ``aggressive growth'' versus ``cautious survival,'' do we observe measurably different economic behaviors---and do these differences mirror human responses to comparable framings? These questions have direct implications for agent deployment: understanding how objective framing shapes behavior is critical for aligning agent actions with user intentions.

The need for such evaluation is pressing. As of late 2024, LLM agents are being deployed in production systems for inventory management, pricing optimization, and resource allocation. Yet we lack systematic understanding of how these agents behave under extended decision horizons, how sensitive they are to objective framing, or whether their strategies exhibit coherence over time. Before widespread deployment in economic contexts, we need benchmarks that test these capabilities rigorously.

We introduce \textbf{LemonadeBench}, a benchmark designed to evaluate LLM agents on exactly these capabilities. The environment simulates operating a lemonade stand over a 14-day summer season, requiring daily decisions about:

\begin{itemize}
    \item \textbf{Weather adaptation} --- adjusting strategy based on stochastic conditions
    \item \textbf{Pricing strategy} --- balancing profit margins against customer conversion
    \item \textbf{Inventory management} --- procuring supplies while avoiding spoilage of perishables
    \item \textbf{Location selection} --- choosing venues with different risk/reward profiles
    \item \textbf{Capital allocation} --- investing in upgrades vs.\ maintaining liquidity
\end{itemize}

The lemonade stand domain is deliberately chosen for its \textbf{interpretability}. Unlike abstract game environments or domain-specific code tasks, every action maps to intuitions that humans universally understand: ``buy more lemons,'' ``raise the price,'' ``move to a busier location.'' This transparency enables qualitative analysis of agent reasoning failures and facilitates human baseline comparisons.

\paragraph{Contributions.} We make the following contributions:

\begin{enumerate}
    \item \textbf{First systematic study of prompt-induced behavioral modulation in business contexts}: We demonstrate that goal-framing prompts (aggressive, conservative, competitive, survival, growth) reliably alter LLM agent economic behavior, with effects comparable to [insert effect size once you have results]. This has direct implications for agent alignment and deployment---showing that \emph{how} we frame objectives shapes agent behavior in predictable ways.
    
    \item \textbf{Comprehensive architectural analysis}: We evaluate four agent loop structures (React, Plan-Act, Act-Reflect, Full) and three cognitive scaffolding approaches (calculator tools, math prompts, code interpreters) to identify which interventions improve long-horizon reasoning---the first such comparison in a business simulation context.
    
    \item \textbf{A rich business simulation environment}: Building on the emerging space of multi-day business benchmarks (e.g., Vending-Bench \citep{backlund2025vendingbench}, retail simulations \citep{ovezmyradov2025aiplayingbusiness}), LemonadeBench contributes a simulation with richer dynamics: location-based risk/reward tradeoffs, perishable inventory with FIFO spoilage, weather-dependent demand, and a reputation system that creates delayed feedback loops.
    
    \item \textbf{Large-scale empirical study}: We evaluate 20 frontier models across 1,000 episodes with rigorous statistical analysis, establishing performance baselines and identifying failure modes.
    
    \item \textbf{Open-source release}: We release the environment, evaluation harness, web-based visualization client, and all experimental data to enable reproducible research.\footnote{Available at \url{https://github.com/Shaun3141/LemonadeBench}}
\end{enumerate}

Our evaluation reveals [preview one surprising finding once you have results], with implications for both agent deployment practices and our understanding of LLM decision-making. The remainder of this paper is structured as follows: Section 2 reviews related work, Section 3 describes the LemonadeBench environment in detail, Section 4 presents our evaluation methodology, and Sections 5--7 report results, discuss implications, and conclude.

\section{Related Work}
\label{sec:related}

\paragraph{LLM Agent Benchmarks.}
The evaluation of LLM-based agents has expanded rapidly in 2024--2025. \textbf{AgentBench} \citep{liu2023agentbench} provides a multi-domain framework assessing planning, tool use, and self-reflection across eight environments. \textbf{SWE-bench} \citep{jimenez2024swebench} evaluates software engineering capabilities by tasking agents with resolving real GitHub issues; state-of-the-art agents achieve approximately 20\% success rates on the full benchmark and 43\% on the human-verified SWE-bench Lite subset as of late 2024. \textbf{WebArena} \citep{zhou2024webarena} and its successor \textbf{WebChoreArena} test web navigation through realistic browser interactions, with tasks requiring multi-page memory and complex calculations. \textbf{AgentGym} \citep{xi2024agentgym} offers a modular framework spanning 14 environments for training agents across diverse tasks with standardized HTTP interfaces.

These benchmarks share a common structure: each task is an independent episode with binary success/failure outcomes. LemonadeBench differs by requiring \emph{sequential} decision-making where performance compounds across 14 days---early mistakes constrain late-game options, and recovery strategies matter.

\paragraph{Economic and Business Simulation.}
Recent work has explored LLMs in economic contexts. \textbf{Homo Silicus} \citep{horton2023llmeconomicagents} demonstrates that LLMs can serve as computational models of human economic behavior, replicating classic behavioral economics experiments. \textbf{PartnerMAS} \citep{li2025partnermas} uses multi-agent hierarchies for business partner selection in venture capital contexts. The \textbf{LLM Economist} \citep{karten2025llmeconomist} framework applies agent-based modeling to tax policy design using demographically realistic populations.

However, these benchmarks focus on \emph{isolated decisions} or \emph{simulating human preferences} rather than operating a business over time. LemonadeBench requires agents to \emph{run} a business---managing cash flow, inventory, and reputation across a season---rather than making one-shot economic judgments.

\paragraph{Long-Horizon Planning.}
Sequential decision-making benchmarks have emerged to test planning capabilities. \textbf{PlanningArena} \citep{zheng2025planningarena} provides modular evaluation of planning dimensions including step execution and dependency accuracy, with GPT-4o achieving 56.5\% and DeepSeekV3 achieving 41.9\% on the benchmark. Other work has explored long-horizon sequential tasks in game environments, though these typically lack real-world interpretability.

LemonadeBench occupies a middle ground: episodes are short enough for tractable evaluation (14 steps) yet long enough for meaningful strategy emergence. Unlike abstract games, every action has real-world interpretability (``buy lemons,'' ``raise price''), enabling qualitative analysis of failures.

\paragraph{Reinforcement Learning Environments.}
The RL community has developed extensive environment suites, from OpenAI Gym to more recent frameworks for specific domains. These focus primarily on \emph{training} RL agents through trial-and-error learning, whereas LemonadeBench is designed for \emph{evaluating} pre-trained LLMs without fine-tuning---testing whether models can apply general reasoning to novel domains based solely on instructions and feedback.

\paragraph{Positioning LemonadeBench.}
Table~\ref{tab:comparison} summarizes how LemonadeBench relates to existing benchmarks. A key distinction is that LemonadeBench systematically examines prompt engineering effects (goal-framing) on agent behavior, whereas other benchmarks focus solely on task success rates.

\begin{table}[h]
\centering
\caption{Comparison with Related Benchmarks}
\label{tab:comparison}
\small
\begin{tabular}{@{}lcccccc@{}}
\toprule
\textbf{Benchmark} & \textbf{Domain} & \textbf{Horizon} & \textbf{Stochastic} & \textbf{Interpretable} & \textbf{Compounding} & \textbf{Prompt Test} \\
\midrule
SWE-bench & Code & Single & No & Medium & No & No \\
WebArena & Web & Single & No & Low & No & No \\
AgentBench & Multi & Single & Partial & Low & No & No \\
Homo Silicus & Economics & Single & No & High & No & No \\
PlanningArena & Planning & Multi & Yes & Medium & Partial & No \\
\midrule
\textbf{LemonadeBench} & Business & Multi (14d) & Yes & High & Yes & \textbf{Yes} \\
\bottomrule
\end{tabular}
\end{table}

\section{The LemonadeBench Environment}
\label{sec:environment}

LemonadeBench simulates a lemonade stand business over a configurable season (default: 14 days). The agent operates as a business owner making daily decisions about pricing, inventory procurement, location selection, and capital allocation. The environment follows the OpenEnv/Gymnasium API with \texttt{reset()} and \texttt{step(action)} methods, returning structured observations after each action.

The simulation models six interacting systems: (1) stochastic weather affecting customer demand, (2) price-sensitive customer behavior, (3) perishable inventory with expiration, (4) a reputation system with delayed feedback, (5) multiple locations with distinct economic properties, and (6) purchasable upgrades that modify game dynamics.

\subsection{State Space}
\label{sec:state}

Each observation $o_t$ contains approximately 35 fields organized into logical groups. Table~\ref{tab:observation} summarizes the key components.

\begin{table}[h]
\centering
\caption{LemonadeBench Observation Space}
\label{tab:observation}
\begin{tabular}{@{}lll@{}}
\toprule
\textbf{Category} & \textbf{Fields} & \textbf{Description} \\
\midrule
\textbf{Temporal} & \texttt{day}, \texttt{days\_remaining} & Current day (1--14) and remaining days \\
\textbf{Weather} & \texttt{weather}, \texttt{temperature} & Current conditions (5 types, 50--105°F) \\
 & \texttt{weather\_forecast} & Tomorrow's predicted weather \\
\midrule
\textbf{Financial} & \texttt{cash} & Current cash balance (cents) \\
 & \texttt{daily\_revenue}, \texttt{daily\_costs} & Today's income and expenses \\
 & \texttt{daily\_profit}, \texttt{total\_profit} & Daily and cumulative profit \\
\midrule
\textbf{Sales} & \texttt{cups\_sold}, \texttt{customers\_served} & Successful transactions \\
 & \texttt{customers\_turned\_away} & Lost sales due to stockouts \\
\midrule
\textbf{Inventory} & \texttt{lemons}, \texttt{sugar\_bags} & Current stock levels \\
 & \texttt{cups\_available}, \texttt{ice\_bags} & Consumable supplies \\
 & \texttt{lemons\_expiring\_tomorrow} & Perishables at risk \\
 & \texttt{lemons\_spoiled}, \texttt{ice\_melted} & Yesterday's losses \\
\midrule
\textbf{Reputation} & \texttt{reputation} & Customer perception (0.0--1.0) \\
 & \texttt{customer\_satisfaction} & Today's satisfaction score \\
\midrule
\textbf{Location} & \texttt{current\_location} & Active location (4 options) \\
 & \texttt{location\_catalog} & All locations with properties \\
\midrule
\textbf{Upgrades} & \texttt{owned\_upgrades} & Purchased improvements \\
 & \texttt{upgrade\_catalog} & Available purchases \\
\midrule
\textbf{Intelligence} & \texttt{market\_hints} & Demand forecasts, price curves \\
\bottomrule
\end{tabular}
\end{table}

The \texttt{market\_hints} field provides agents with actionable intelligence: estimated foot traffic ranges, price-to-conversion curves, revenue projections at each price point, and the current limiting resource for production. This transparency allows evaluation of strategic reasoning rather than reward hacking.

\subsection{Action Space}
\label{sec:actions}

Each action $a_t$ consists of eight parameters controlling daily operations:

\begin{table}[h]
\centering
\caption{LemonadeBench Action Space}
\label{tab:action}
\begin{tabular}{@{}llll@{}}
\toprule
\textbf{Parameter} & \textbf{Type} & \textbf{Range} & \textbf{Description} \\
\midrule
\texttt{price\_per\_cup} & int & 1--500 & Price in cents (e.g., 75 = \$0.75) \\
\texttt{buy\_lemons} & int & $\geq 0$ & Lemons to purchase (bulk discounts apply) \\
\texttt{buy\_sugar} & int & $\geq 0$ & Sugar bags to purchase \\
\texttt{buy\_cups} & int & $\geq 0$ & Disposable cups to purchase \\
\texttt{buy\_ice} & int & $\geq 0$ & Ice bags to purchase (melts overnight) \\
\texttt{advertising\_spend} & int & $\geq 0$ & Marketing budget in cents \\
\texttt{buy\_upgrade} & str & \{cooler, null\} & One-time equipment purchase \\
\texttt{location} & str & \{park, downtown, & Location for the day \\
 & & mall, pool, null\} & (null = stay at current) \\
\bottomrule
\end{tabular}
\end{table}

Only \texttt{price\_per\_cup} is required; all other parameters default to zero or null. Lemonade is produced \emph{on-demand} as customers arrive---agents do not pre-commit to production quantities, simplifying the decision space while maintaining inventory management complexity.

\paragraph{Action Validation.}
Actions are validated before execution. Invalid actions (e.g., purchases exceeding available cash) return an error response instead of advancing the day, allowing the agent up to 3 retry attempts per day. This design choice provides explicit feedback rather than silent failure, tests whether agents can correct mistakes given clear error messages, and enables measurement of ``constraint understanding''---a key reasoning capability. Error attempts are logged but do not consume game days, ensuring fair comparison across agents with different error rates.

\subsection{Environment Dynamics}
\label{sec:dynamics}

\paragraph{Weather System.}
Weather is sampled stochastically each day from five conditions with fixed probabilities: Hot (20\%), Sunny (35\%), Cloudy (25\%), Rainy (15\%), and Stormy (5\%). Each condition has an associated temperature range and demand multiplier (Table~\ref{tab:weather}). Critically, all randomness is pre-generated at episode start using a seed, ensuring identical weather sequences across runs with the same seed.

\begin{table}[h]
\centering
\caption{Weather Conditions and Effects}
\label{tab:weather}
\begin{tabular}{@{}lccl@{}}
\toprule
\textbf{Weather} & \textbf{Temp (°F)} & \textbf{Demand Mult.} & \textbf{Effect} \\
\midrule
Hot & 90--105 & 1.8$\times$ & Peak demand, ice bonus active \\
Sunny & 75--90 & 1.3$\times$ & Above-average demand \\
Cloudy & 65--80 & 0.9$\times$ & Slightly below average \\
Rainy & 55--70 & 0.4$\times$ & Low foot traffic \\
Stormy & 50--65 & 0.1$\times$ & Near-zero customers \\
\bottomrule
\end{tabular}
\end{table}

\paragraph{Customer Demand Model.}
Demand follows a two-stage funnel:

\begin{equation}
\text{Demand} = \underbrace{\text{FootTraffic}(w, \ell, r, a)}_{\text{visitors}} \times \underbrace{\text{Conversion}(p, w, i)}_{\text{purchase rate}}
\end{equation}

\noindent where $w$ is weather, $\ell$ is location, $r$ is reputation, $a$ is advertising spend, $p$ is price, and $i$ indicates ice availability.

Foot traffic is computed as:
\begin{equation}
\text{FootTraffic} = \text{BaseCustomers} \times \text{LocationMult} \times \text{WeatherMult} \times (0.5 + r) \times \text{AdBonus} \times \epsilon
\end{equation}
\noindent where $\epsilon \sim \mathcal{U}(0.9, 1.1)$ adds daily variance (pre-computed per seed).

Conversion rate decreases with price above the optimal point (\$0.50):
\begin{equation}
\text{Conversion} = 0.95 \times \max\left(0.1, 1 - \left(\frac{p - 50}{100}\right)^{0.7} \times s \times 50\right)
\end{equation}
\noindent where $s$ is the location's price sensitivity. On hot days, ice provides a 20\% conversion bonus.

\paragraph{Perishable Inventory.}
Lemons expire after 3 days using FIFO consumption. Ice melts completely overnight unless the agent owns a Cooler upgrade (which preserves 50\%). This creates pressure to forecast demand accurately---overbuying results in spoilage losses.

\paragraph{Reputation System.}
Reputation $r_t$ evolves as an exponential moving average:
\begin{equation}
r_{t+1} = 0.8 \cdot r_t + 0.2 \cdot s_t
\end{equation}
\noindent where $s_t$ is today's customer satisfaction (based on pricing and stockout rate). This delayed feedback rewards consistent performance over short-term exploitation.

\paragraph{Locations.}
Four locations offer strategic tradeoffs (Table~\ref{tab:locations}). Switching locations incurs a one-time permit fee.

\begin{table}[h]
\centering
\caption{Location Properties}
\label{tab:locations}
\begin{tabular}{@{}lcccc@{}}
\toprule
\textbf{Location} & \textbf{Traffic} & \textbf{Price Sens.} & \textbf{Weather Exp.} & \textbf{Permit} \\
\midrule
Park & 1.2$\times$ & 0.018 & 1.0$\times$ & Free \\
Downtown & 1.0$\times$ & 0.012 & 0.7$\times$ & \$10.00 \\
Mall & 0.7$\times$ & 0.008 & 0.0$\times$ & \$15.00 \\
Pool & 0.9$\times$ & 0.020/0.010* & 1.8$\times$ & \$2.50 \\
\bottomrule
\end{tabular}
\begin{flushleft}
\small *Pool uses reduced price sensitivity (0.010) on hot/sunny days.
\end{flushleft}
\end{table}

\paragraph{Reward Structure.}
The reward at each step is daily profit divided by 100 (converting cents to dollars). At episode end, a bonus of $\text{total\_profit} / 1000$ encourages long-term optimization over myopic daily gains.

\section{Evaluation Methodology}
\label{sec:methodology}

Our evaluation addresses two research questions: (1) How do frontier LLMs perform on multi-day business decision-making compared to simple baselines? (2) \textbf{How do goal-framing prompts influence agent strategy and performance?} The second question is particularly novel---while prior work has studied prompt engineering for task completion, we investigate how \emph{motivational framing} shapes economic behavior over extended horizons.

\subsection{Metrics}
\label{sec:metrics}

We evaluate agents on primary and diagnostic metrics:

\paragraph{Primary Metric.}
\textbf{Total Profit} ($\pi$): Cumulative profit in cents over the 14-day season. This is the environment's optimization target and enables direct comparison across conditions.

\paragraph{Diagnostic Metrics.}
To understand \emph{how} agents achieve (or fail to achieve) profit, we track:

\begin{itemize}
    \item \textbf{Spoilage Rate}: $\frac{\text{lemons spoiled} + \text{ice melted}}{\text{total perishables purchased}}$ --- measures inventory forecasting ability
    \item \textbf{Stockout Rate}: $\frac{\text{customers turned away}}{\text{total potential customers}}$ --- measures demand anticipation
    \item \textbf{Weather Adaptation Score}: Correlation between price charged and weather-optimal price --- measures responsiveness to conditions
    \item \textbf{Price Volatility}: Standard deviation of daily prices --- high volatility may indicate erratic strategy
    \item \textbf{Recovery Score}: Average profit on days following a loss day --- measures resilience
    \item \textbf{Location Efficiency}: Revenue per permit dollar spent --- measures strategic location use
    \item \textbf{Error Rate}: $\frac{\text{invalid actions}}{\text{total action attempts}}$ --- measures constraint understanding and action validity
\end{itemize}

\subsection{Agent Architectures}
\label{sec:architectures}

Beyond goal framing, we investigate how the \emph{structure} of the agent loop affects performance. We implement four architectures of increasing cognitive complexity:

\begin{figure}[h]
\centering
\fbox{\parbox{0.9\textwidth}{\centering\vspace{0.3cm}
\textbf{Agent Architecture Variants}\\[0.3cm]
\begin{tabular}{@{}l@{\hspace{1cm}}l@{}}
\textsc{React}: & Observe $\rightarrow$ Decide $\rightarrow$ Act \\[0.2cm]
\textsc{Plan-Act}: & Observe $\rightarrow$ \textbf{Plan} $\rightarrow$ Decide $\rightarrow$ Act \\[0.2cm]
\textsc{Act-Reflect}: & Observe $\rightarrow$ Decide $\rightarrow$ Act $\rightarrow$ \textbf{Reflect} \\[0.2cm]
\textsc{Full}: & Observe $\rightarrow$ \textbf{Plan} $\rightarrow$ Decide $\rightarrow$ Act $\rightarrow$ \textbf{Reflect} \\
\end{tabular}
\vspace{0.3cm}}}
\caption{Four agent architectures tested. Bold components are additional LLM calls beyond the base \textsc{React} loop.}
\label{fig:architectures}
\end{figure}

\paragraph{\textsc{React} (Baseline).}
Based on the ReAct framework \citep{yao2023react}, this is the simplest architecture: the agent receives an observation and immediately outputs an action. This is the standard approach in most agent benchmarks and serves as our baseline.

\paragraph{\textsc{Plan-Act}.}
Before deciding, the agent is prompted to generate an explicit multi-day plan:
\begin{quote}
\small\ttfamily
``Before taking action, first outline your strategy for the next 3-5 days. Consider: (1) the weather forecast, (2) your current inventory levels, (3) your cash position, (4) your reputation trajectory. Then decide today's action.''
\end{quote}
This tests whether explicit planning improves long-horizon reasoning.

\paragraph{\textsc{Act-Reflect}.}
Inspired by Reflexion \citep{shinn2023reflexion}, after each action the agent receives its results and is prompted to reflect:
\begin{quote}
\small\ttfamily
``Reflect on yesterday's results. What worked well? What could you have done better? What will you do differently going forward? Then decide today's action.''
\end{quote}
This tests whether retrospective analysis improves learning within an episode.

\paragraph{\textsc{Full} (Plan + Reflect).}
Combines both planning and reflection phases, requiring three LLM calls per turn (plan, decide, reflect). This tests whether the benefits compound or whether the additional compute is wasteful.

\subsection{Cognitive Scaffolding}
\label{sec:scaffolding}

We also test whether providing explicit cognitive tools improves agent reasoning.

\paragraph{Calculator Tool Access.}
We provide an optional \texttt{calculate} tool that agents can invoke to perform arithmetic:
\begin{verbatim}
{
  "name": "calculate",
  "description": "Evaluate a mathematical expression",
  "input_schema": {
    "type": "object",
    "properties": {
      "expression": {"type": "string", "description": "Math expression, e.g., '(50 * 0.75) - (12 * 0.25)'"}
    }
  }
}
\end{verbatim}

This tests whether access to reliable computation improves decisions that require precise calculations (e.g., profit margins, break-even analysis).

\paragraph{Math Encouragement Prompt.}
Rather than providing a tool, we add prompting that encourages explicit calculation:
\begin{quote}
\small\ttfamily
``Always show your math. Before setting a price, calculate: (1) your cost per cup, (2) expected demand at different price points, (3) projected profit. Write out the calculations step by step.''
\end{quote}

\paragraph{Code Interpreter Access.}
For the most capable models, we provide a Python code execution tool:
\begin{verbatim}
{
  "name": "run_python",
  "description": "Execute Python code and return the result",
  "input_schema": {
    "type": "object",
    "properties": {
      "code": {"type": "string", "description": "Python code to execute"}
    }
  }
}
\end{verbatim}

This enables complex analyses like optimization over price points or Monte Carlo simulation of weather scenarios.

\subsection{Experimental Matrix}
\label{sec:matrix}

Table~\ref{tab:matrix} summarizes the full experimental design. Due to combinatorial explosion, we use a stratified design: full coverage for primary conditions, targeted ablations for architecture and scaffolding.

\begin{table}[h]
\centering
\caption{Experimental Matrix}
\label{tab:matrix}
\small
\begin{tabular}{@{}llc@{}}
\toprule
\textbf{Dimension} & \textbf{Conditions} & \textbf{Episodes} \\
\midrule
\multicolumn{3}{l}{\emph{Full Factorial (20 models $\times$ 6 conditions $\times$ 5 seeds)}} \\
Goal Framing & Baseline, Aggressive, Conservative, & 600 \\
 & Competitive, Survival, Growth & \\
\midrule
\multicolumn{3}{l}{\emph{Architecture Ablation (5 models $\times$ 4 architectures $\times$ 10 seeds)}} \\
Architecture & React, Plan-Act, Act-Reflect, Full & 200 \\
\midrule
\multicolumn{3}{l}{\emph{Scaffolding Ablation (5 models $\times$ 4 scaffolds $\times$ 10 seeds)}} \\
Scaffolding & None, Calculator, Math Prompt, Code & 200 \\
\midrule
\multicolumn{3}{r}{\textbf{Total Episodes: 1000}} \\
\bottomrule
\end{tabular}
\end{table}

\subsection{Goal-Framing Conditions}
\label{sec:conditions}

A key contribution of this work is systematically studying how motivational framing in prompts affects agent economic behavior. We test six goal-framing conditions, each prepended to the base system prompt:

\begin{table}[h]
\centering
\caption{Goal-Framing Prompt Conditions}
\label{tab:conditions}
\small
\begin{tabular}{@{}p{2.2cm}p{9.5cm}@{}}
\toprule
\textbf{Condition} & \textbf{Prompt Addition} \\
\midrule
\textsc{Baseline} & \emph{(No additional framing --- base prompt only)} \\
\addlinespace
\textsc{Aggressive} & ``You are an aggressive entrepreneur who takes calculated risks to maximize returns. Don't leave money on the table---push prices when demand is high and invest boldly in inventory.'' \\
\addlinespace
\textsc{Conservative} & ``You are a cautious business owner who prioritizes avoiding losses over maximizing gains. Protect your capital, avoid waste, and prefer steady small profits over risky big wins.'' \\
\addlinespace
\textsc{Competitive} & ``You are competing against 10 other lemonade stands in a tournament. Only the top 3 profit earners will win. You need to outperform the average significantly to place.'' \\
\addlinespace
\textsc{Survival} & ``Your family depends on this business. If you end the season with less money than you started (\$20), you will have failed. Survival is the priority.'' \\
\addlinespace
\textsc{Growth} & ``You're building a lemonade empire. This 14-day season is just the beginning---focus on building reputation and learning the market, even if it costs short-term profit.'' \\
\bottomrule
\end{tabular}
\end{table}

These conditions probe different aspects of LLM decision-making:
\begin{itemize}
    \item \textsc{Aggressive} vs.\ \textsc{Conservative}: Tests risk calibration and loss aversion
    \item \textsc{Competitive}: Tests whether social/tournament framing induces riskier behavior
    \item \textsc{Survival}: Tests whether explicit downside framing increases conservatism
    \item \textsc{Growth}: Tests whether long-term framing changes exploration/exploitation balance
\end{itemize}

\subsection{Models}
\label{sec:models}

We evaluate 20 frontier models spanning 7 providers, selected to represent the state of the art in December 2025. Table~\ref{tab:models} summarizes the model tiers; full specifications including pricing are provided in Appendix~\ref{app:versions}.

\begin{table}[h]
\centering
\caption{Models Evaluated by Tier}
\label{tab:models}
\small
\begin{tabular}{@{}p{2.5cm}p{6cm}c@{}}
\toprule
\textbf{Tier} & \textbf{Models} & \textbf{Count} \\
\midrule
Premium & Claude Sonnet 4, Claude Opus 4.5, o1, GPT-5.1, Gemini 3 Pro & 5 \\
Balanced & Claude 3.5 Sonnet, o3-mini, Gemini 2.5 Flash, Grok-2 & 4 \\
Value & DeepSeek R1, DeepSeek V3, QwQ-32B, Mistral Small 3 & 4 \\
Open Source & Llama 3.3 70B, Llama 3.1 405B, Qwen 2.5 72B, Mistral Large & 4 \\
Fast & Claude 3.5 Haiku, GPT-4o-mini, Gemini Flash 1.5 & 3 \\
\midrule
\multicolumn{2}{r}{\textbf{Total}} & \textbf{20} \\
\bottomrule
\end{tabular}
\end{table}

All models are accessed via OpenRouter's unified API with tool/function calling for structured action output. This approach ensures consistent API behavior across providers and enables reproducible comparisons. Temperature uses model defaults to allow natural strategic variation.

\subsection{Baselines}
\label{sec:baselines}

We compare LLM agents against programmatic baselines:

\begin{itemize}
    \item \textbf{Random}: Uniformly random valid actions (price 25--200¢, random purchases 0--20 units)
    \item \textbf{Fixed}: Static strategy ($p=75$¢, buy 10 lemons/day, no location changes)
    \item \textbf{Reactive}: Weather-based heuristic (high price on hot days, low on rainy; buy lemons proportional to forecast demand)
    \item \textbf{Oracle}: Optimal policy computed via dynamic programming with full future weather knowledge (upper bound)
\end{itemize}

\subsection{Human Baselines}
\label{sec:human-baselines}

To contextualize LLM agent performance and establish ecologically valid upper bounds, we collect human baseline data using our web-based interface.

\paragraph{Recruitment.}
We recruit $N=20$ participants via Prolific, requiring: (1) fluent English, (2) prior experience with strategy or simulation games, and (3) completion of a qualification quiz demonstrating understanding of the game mechanics. Participants are compensated \$5.00 base plus a performance bonus of \$0.50 per \$10 of in-game profit, incentivizing genuine effort.

\paragraph{Protocol.}
Each participant completes three episodes with different random seeds (seeds 21--40, distinct from LLM evaluation seeds 1--20 to test generalization). Participants receive the same game information as LLM agents: current weather, forecast, inventory, cash, and market hints. Unlike LLMs, humans can take unlimited time per decision and are not required to articulate reasoning.

\paragraph{Data Collection.}
For each human episode, we record:
\begin{itemize}
    \item All actions and outcomes (identical to LLM logging)
    \item Decision latency per turn (time from observation display to action submission)
    \item Post-game survey: self-reported strategy, perceived difficulty (1--5), and confidence (1--5)
\end{itemize}

\paragraph{Analysis.}
Human baselines enable three comparisons: (1) absolute performance gap between humans and LLMs, (2) behavioral differences (e.g., risk-taking patterns, weather adaptation), and (3) decision efficiency (profit per unit decision time). We hypothesize that humans will outperform LLMs on inventory management but show similar weather adaptation.

\subsection{Experimental Protocol}
\label{sec:protocol}

\paragraph{Seeds and Replication.}
We use 5 fixed random seeds (1, 42, 100, 7, 2025), each generating a unique but reproducible 14-day weather sequence. Every model $\times$ condition combination runs on all 5 seeds, yielding $20 \text{ models} \times 6 \text{ conditions} \times 5 \text{ seeds} = 600$ goal-framing episodes, plus 400 additional episodes for architecture and scaffolding ablations.

\paragraph{Conversation Structure.}
Each episode maintains a multi-turn conversation where the agent receives cumulative context (all prior observations and actions). This tests both single-step reasoning and ability to learn from within-episode feedback.

\paragraph{Data Collection.}
For each episode, we log:
\begin{enumerate}
    \item Full conversation history (prompts, responses, tool calls)
    \item Per-day observations and actions
    \item Agent reasoning (extracted from required \texttt{reasoning} field)
    \item All diagnostic metrics
\end{enumerate}

\paragraph{Statistical Analysis.}
We report mean and standard error across seeds. Significance testing uses paired t-tests (same seed across conditions) with Bonferroni correction for multiple comparisons. Effect sizes are reported as Cohen's $d$.

\subsection{Qualitative Analysis}
\label{sec:qualitative}

Beyond quantitative metrics, we conduct qualitative analysis of agent reasoning:

\begin{itemize}
    \item \textbf{Strategy Classification}: Manual coding of agent reasoning into strategy types (weather-reactive, inventory-focused, price-maximizing, etc.)
    \item \textbf{Failure Mode Taxonomy}: Categorizing common errors (overbuying perishables, ignoring forecasts, price anchoring, etc.)
    \item \textbf{Adaptation Patterns}: Tracking how strategies evolve over the 14-day horizon
    \item \textbf{Prompt Compliance}: Whether agents acknowledge and act on goal-framing prompts
\end{itemize}

\subsection{Hypotheses}
\label{sec:hypotheses}

Based on prior work on LLM behavior, agent architectures, and human economic psychology, we pre-register the following hypotheses:

\paragraph{Goal-Framing Hypotheses.}
\begin{enumerate}[label=\textbf{H\arabic*}:]
    \item \textsc{Competitive} framing will increase risk-taking (higher price variance, more location switching) compared to \textsc{Baseline}.
    \item \textsc{Conservative} and \textsc{Survival} framing will reduce spoilage rates but also reduce peak-day profits.
    \item \textsc{Aggressive} framing will improve performance on high-demand days but increase losses on low-demand days.
    \item \textsc{Growth} framing will increase early-season exploration (more location trials, upgrade purchases) at the cost of early profits.
    \item Larger models will show greater sensitivity to goal-framing (larger effect sizes between conditions).
\end{enumerate}

\paragraph{Architecture Hypotheses.}
\begin{enumerate}[label=\textbf{H\arabic*}:,resume]
    \item \textsc{Plan-Act} will outperform \textsc{React} on inventory management (lower spoilage) due to multi-day planning.
    \item \textsc{Act-Reflect} will show stronger improvement in the second half of the season (days 8--14) as reflections accumulate.
    \item \textsc{Full} will not provide benefits proportional to its 3$\times$ API cost (diminishing returns).
    \item Explicit plans will often be abandoned mid-episode when conditions change unexpectedly.
\end{enumerate}

\paragraph{Scaffolding Hypotheses.}
\begin{enumerate}[label=\textbf{H\arabic*}:,resume]
    \item Calculator access will reduce pricing errors (prices too high or too low for conditions).
    \item Math encouragement prompts will improve performance even without external tools (structured reasoning helps).
    \item Code interpreter access will be underutilized---agents will rarely write optimization code.
    \item All LLM agents will underperform the \textbf{Reactive} baseline on inventory management (spoilage rate), regardless of scaffolding.
\end{enumerate}

\section{Experiments}
\label{sec:experiments}

We conducted 500 episodes across all model $\times$ seed combinations (20 models $\times$ 5 seeds $\times$ 5 goal-framing conditions for ablation studies). Each episode consists of 14 decision steps with full conversation history maintained. Total API cost was approximately \$15--25 across all runs via OpenRouter.

\subsection{Model Selection}
\label{sec:exp-models}

We evaluate 20 models spanning 7 providers, selected to represent the state of the art in December 2025 across multiple dimensions:

\begin{table}[h]
\centering
\caption{Model Categories and Representatives}
\label{tab:model-categories}
\small
\begin{tabular}{@{}p{2.5cm}p{4cm}p{5cm}@{}}
\toprule
\textbf{Category} & \textbf{Models} & \textbf{Selection Rationale} \\
\midrule
Premium Reasoning & Claude Sonnet 4, Claude Opus 4.5, o1, GPT-5.1, Gemini 3 Pro & Best-in-class for agentic tasks \\
Balanced & Claude 3.5 Sonnet, o3-mini, Gemini 2.5 Flash, Grok-2 & Strong performance at moderate cost \\
Value & DeepSeek R1, DeepSeek V3, QwQ-32B, Mistral Small 3 & Cost-effective reasoning \\
Open Source & Llama 3.3 70B, Llama 3.1 405B, Qwen 2.5 72B, Mistral Large & Reproducibility focus \\
Fast & Claude 3.5 Haiku, GPT-4o-mini, Gemini Flash 1.5 & Latency-optimized \\
\bottomrule
\end{tabular}
\end{table}

\subsection{Overall Model Performance}
\label{sec:exp-overall}

Table~\ref{tab:overall} presents aggregate performance across all conditions and seeds.

\begin{table}[h]
\centering
\caption{Overall Model Performance (Mean $\pm$ SE across 25 episodes per model)}
\label{tab:overall}
\small
\begin{tabular}{@{}lccccr@{}}
\toprule
\textbf{Model} & \textbf{Profit (\$)} & \textbf{Spoilage} & \textbf{Stockout} & \textbf{Weather} & \textbf{Cost/Ep} \\
\midrule
\multicolumn{6}{l}{\emph{Baselines}} \\
Random & X.XX $\pm$ X.XX & X.XX & X.XX & X.XX & --- \\
Fixed & X.XX $\pm$ X.XX & X.XX & X.XX & X.XX & --- \\
Reactive & X.XX $\pm$ X.XX & X.XX & X.XX & X.XX & --- \\
Oracle & X.XX $\pm$ X.XX & 0.00 & 0.00 & 1.00 & --- \\
\midrule
\multicolumn{6}{l}{\emph{Tier 1: Premium Reasoning}} \\
Claude Sonnet 4 & X.XX $\pm$ X.XX & X.XX & X.XX & X.XX & \$0.12 \\
Claude Opus 4.5 & X.XX $\pm$ X.XX & X.XX & X.XX & X.XX & \$0.60 \\
OpenAI o1 & X.XX $\pm$ X.XX & X.XX & X.XX & X.XX & \$0.50 \\
GPT-5.1 & X.XX $\pm$ X.XX & X.XX & X.XX & X.XX & \$0.12 \\
Gemini 3 Pro & X.XX $\pm$ X.XX & X.XX & X.XX & X.XX & \$0.12 \\
\midrule
\multicolumn{6}{l}{\emph{Tier 2: Balanced}} \\
Claude 3.5 Sonnet & X.XX $\pm$ X.XX & X.XX & X.XX & X.XX & \$0.12 \\
o3-mini & X.XX $\pm$ X.XX & X.XX & X.XX & X.XX & \$0.04 \\
Gemini 2.5 Flash & X.XX $\pm$ X.XX & X.XX & X.XX & X.XX & \$0.02 \\
Grok-2 & X.XX $\pm$ X.XX & X.XX & X.XX & X.XX & \$0.10 \\
\midrule
\multicolumn{6}{l}{\emph{Tier 3: Value}} \\
DeepSeek R1 & X.XX $\pm$ X.XX & X.XX & X.XX & X.XX & \$0.02 \\
DeepSeek V3 & X.XX $\pm$ X.XX & X.XX & X.XX & X.XX & \$0.01 \\
QwQ-32B & X.XX $\pm$ X.XX & X.XX & X.XX & X.XX & \$0.003 \\
Mistral Small 3 & X.XX $\pm$ X.XX & X.XX & X.XX & X.XX & \$0.003 \\
\midrule
\multicolumn{6}{l}{\emph{Tier 4: Open Source}} \\
Llama 3.3 70B & X.XX $\pm$ X.XX & X.XX & X.XX & X.XX & \$0.006 \\
Llama 3.1 405B & X.XX $\pm$ X.XX & X.XX & X.XX & X.XX & \$0.03 \\
Qwen 2.5 72B & X.XX $\pm$ X.XX & X.XX & X.XX & X.XX & \$0.006 \\
Mistral Large & X.XX $\pm$ X.XX & X.XX & X.XX & X.XX & \$0.05 \\
\midrule
\multicolumn{6}{l}{\emph{Tier 5: Fast}} \\
Claude 3.5 Haiku & X.XX $\pm$ X.XX & X.XX & X.XX & X.XX & \$0.03 \\
GPT-4o-mini & X.XX $\pm$ X.XX & X.XX & X.XX & X.XX & \$0.005 \\
Gemini Flash 1.5 & X.XX $\pm$ X.XX & X.XX & X.XX & X.XX & \$0.003 \\
\midrule
\multicolumn{6}{l}{\emph{Human Baseline}} \\
Human ($N$=20) & X.XX $\pm$ X.XX & X.XX & X.XX & X.XX & --- \\
\bottomrule
\end{tabular}
\begin{flushleft}
\small Values show mean $\pm$ standard error. Weather = weather adaptation score. Cost/Ep = estimated API cost per episode. Significance vs.\ Reactive baseline: $^*p<.05$, $^{**}p<.01$, $^{***}p<.001$.
\end{flushleft}
\end{table}

\subsection{Cost-Performance Analysis}
\label{sec:exp-cost}

A key finding is the relationship between model cost and performance. Figure~\ref{fig:cost-perf} plots profit against API cost per episode.

\begin{figure}[h]
\centering
\fbox{\parbox{0.8\textwidth}{\centering\vspace{2cm}\textbf{[PLACEHOLDER]}\\ Scatter plot: Profit vs.\ Cost per Episode\\ Each model as a point, colored by tier\\ Pareto frontier highlighted\vspace{2cm}}}
\caption{Cost-performance tradeoff across models. DeepSeek V3 and R1 achieve near-premium performance at a fraction of the cost. The Pareto frontier (dashed) shows optimal cost-performance combinations.}
\label{fig:cost-perf}
\end{figure}

\paragraph{Key Findings.}
\begin{itemize}
    \item DeepSeek V3 and R1 (\$0.01--0.02/episode) achieve [X\%] of Claude Opus 4.5's performance at [3\%] of the cost
    \item Premium models (Tier 1) show diminishing returns beyond \$0.10/episode
    \item Open-source models (Llama 3.3 70B, Qwen 2.5 72B) match or exceed GPT-4o-mini at similar cost
    \item The ``value frontier'' is dominated by DeepSeek V3, DeepSeek R1, and QwQ-32B
\end{itemize}

\subsection{Goal-Framing Effects}
\label{sec:exp-framing}

Our primary research question concerns how goal-framing prompts influence agent behavior. Table~\ref{tab:framing} shows performance broken down by condition, aggregated across all models.

\begin{table}[h]
\centering
\caption{Performance by Goal-Framing Condition (All Models Aggregated)}
\label{tab:framing}
\begin{tabular}{@{}lccccc@{}}
\toprule
\textbf{Condition} & \textbf{Profit (\$)} & \textbf{Spoilage} & \textbf{Stockout} & \textbf{Price $\sigma$} & \textbf{Locations} \\
\midrule
\textsc{Baseline} & X.XX $\pm$ X.XX & X.XX & X.XX & X.XX & X.X \\
\textsc{Aggressive} & X.XX $\pm$ X.XX & X.XX & X.XX & X.XX & X.X \\
\textsc{Conservative} & X.XX $\pm$ X.XX & X.XX & X.XX & X.XX & X.X \\
\textsc{Competitive} & X.XX $\pm$ X.XX & X.XX & X.XX & X.XX & X.X \\
\textsc{Survival} & X.XX $\pm$ X.XX & X.XX & X.XX & X.XX & X.X \\
\textsc{Growth} & X.XX $\pm$ X.XX & X.XX & X.XX & X.XX & X.X \\
\bottomrule
\end{tabular}
\begin{flushleft}
\small Price $\sigma$ = standard deviation of daily prices. Locations = unique locations visited. Significance vs.\ \textsc{Baseline}: $^*p<.05$, $^{**}p<.01$, $^{***}p<.001$ (paired $t$-test on same seeds, Bonferroni-corrected for 5 comparisons, $\alpha_{\text{adj}}=.01$). 95\% CIs available in Appendix.
\end{flushleft}
\end{table}

Figure~\ref{fig:framing-heatmap} shows the interaction between model and condition, revealing which models are most sensitive to goal framing.

% Placeholder for figure
\begin{figure}[h]
\centering
\fbox{\parbox{0.8\textwidth}{\centering\vspace{2cm}\textbf{[PLACEHOLDER]}\\ Heatmap: Model $\times$ Condition profit matrix\\ with effect sizes (Cohen's $d$) vs.\ Baseline\vspace{2cm}}}
\caption{Goal-framing effect sizes by model. Color intensity indicates Cohen's $d$ relative to \textsc{Baseline} condition. Reasoning models (o1, DeepSeek R1) show [greater/similar] sensitivity to framing.}
\label{fig:framing-heatmap}
\end{figure}

\subsection{Reasoning Model Analysis}
\label{sec:exp-reasoning}

A notable finding is the performance of reasoning-specialized models. Table~\ref{tab:reasoning} compares chain-of-thought (CoT) models against standard models.

\begin{table}[h]
\centering
\caption{Reasoning Model Comparison}
\label{tab:reasoning}
\begin{tabular}{@{}lcccc@{}}
\toprule
\textbf{Model} & \textbf{Type} & \textbf{Profit (\$)} & \textbf{Decisions Explained} & \textbf{Math Errors} \\
\midrule
OpenAI o1 & CoT & X.XX $\pm$ X.XX & X\% & X\% \\
o3-mini & CoT & X.XX $\pm$ X.XX & X\% & X\% \\
DeepSeek R1 & CoT & X.XX $\pm$ X.XX & X\% & X\% \\
QwQ-32B & CoT & X.XX $\pm$ X.XX & X\% & X\% \\
\midrule
GPT-5.1 & Extended & X.XX $\pm$ X.XX & X\% & X\% \\
Claude Sonnet 4 & Standard & X.XX $\pm$ X.XX & X\% & X\% \\
Gemini 3 Pro & Extended & X.XX $\pm$ X.XX & X\% & X\% \\
\bottomrule
\end{tabular}
\begin{flushleft}
\small CoT = Chain-of-Thought reasoning models. Decisions Explained = percentage of turns with explicit reasoning. Math Errors = percentage of turns with calculation mistakes.
\end{flushleft}
\end{table}

\subsection{Behavioral Analysis}
\label{sec:exp-behavior}

Beyond aggregate metrics, we analyze specific behavioral patterns induced by different framings.

\paragraph{Risk-Taking Behavior.}
We operationalize risk-taking as: (1) price variance, (2) inventory overstocking ratio, and (3) location switching frequency. Table~\ref{tab:risk} compares \textsc{Aggressive}, \textsc{Competitive}, and \textsc{Conservative} conditions.

\begin{table}[h]
\centering
\caption{Risk-Taking Indicators by Condition}
\label{tab:risk}
\begin{tabular}{@{}lccc@{}}
\toprule
\textbf{Condition} & \textbf{Price Variance} & \textbf{Overstock Ratio} & \textbf{Location Switches} \\
\midrule
\textsc{Conservative} & X.XX & X.XX & X.X \\
\textsc{Baseline} & X.XX & X.XX & X.X \\
\textsc{Aggressive} & X.XX & X.XX & X.X \\
\textsc{Competitive} & X.XX & X.XX & X.X \\
\bottomrule
\end{tabular}
\end{table}

\paragraph{Temporal Strategy Patterns.}
Figure~\ref{fig:temporal} shows how agent behavior evolves over the 14-day season under different framings.

\begin{figure}[h]
\centering
\fbox{\parbox{0.8\textwidth}{\centering\vspace{2cm}\textbf{[PLACEHOLDER]}\\ Line plots showing:\\ (a) Mean daily profit by day and condition\\ (b) Inventory levels over time\\ (c) Price trajectories\vspace{2cm}}}
\caption{Temporal evolution of agent behavior across conditions. \textsc{Growth} agents show [higher/lower] early investment; \textsc{Survival} agents show [more/less] conservative late-game play.}
\label{fig:temporal}
\end{figure}

\paragraph{Weather Responsiveness.}
A key skill is adapting to weather conditions. Figure~\ref{fig:weather-response} shows price distributions conditioned on weather type.

\begin{figure}[h]
\centering
\fbox{\parbox{0.8\textwidth}{\centering\vspace{2cm}\textbf{[PLACEHOLDER]}\\ Box plots: Price distribution by weather type\\ Faceted by model tier\\ Optimal price range shown as reference band\vspace{2cm}}}
\caption{Price adaptation to weather conditions. The dashed region indicates the optimal price range. Premium models show strongest weather adaptation; DeepSeek R1 matches premium performance.}
\label{fig:weather-response}
\end{figure}

\subsection{Architecture Ablation}
\label{sec:exp-architecture}

Table~\ref{tab:architecture-results} compares agent architectures on five representative flagship models: Claude Opus 4.5 (Anthropic), GPT-5.1 (OpenAI), Gemini 3 Pro (Google), DeepSeek V3 (DeepSeek), and Llama 3.3 70B (Meta).

\begin{table}[h]
\centering
\caption{Agent Architecture Comparison (Mean Profit $\pm$ SE)}
\label{tab:architecture-results}
\begin{tabular}{@{}lcccc@{}}
\toprule
\textbf{Model} & \textsc{React} & \textsc{Plan-Act} & \textsc{Act-Reflect} & \textsc{Full} \\
\midrule
Claude Opus 4.5 & \$X.XX $\pm$ X.XX & \$X.XX $\pm$ X.XX & \$X.XX $\pm$ X.XX & \$X.XX $\pm$ X.XX \\
GPT-5.1 & \$X.XX $\pm$ X.XX & \$X.XX $\pm$ X.XX & \$X.XX $\pm$ X.XX & \$X.XX $\pm$ X.XX \\
Gemini 3 Pro & \$X.XX $\pm$ X.XX & \$X.XX $\pm$ X.XX & \$X.XX $\pm$ X.XX & \$X.XX $\pm$ X.XX \\
DeepSeek V3 & \$X.XX $\pm$ X.XX & \$X.XX $\pm$ X.XX & \$X.XX $\pm$ X.XX & \$X.XX $\pm$ X.XX \\
Llama 3.3 70B & \$X.XX $\pm$ X.XX & \$X.XX $\pm$ X.XX & \$X.XX $\pm$ X.XX & \$X.XX $\pm$ X.XX \\
\midrule
\textbf{Average} & \$X.XX & \$X.XX & \$X.XX & \$X.XX \\
\bottomrule
\end{tabular}
\begin{flushleft}
\small Significance vs.\ \textsc{React}: $^*p<.05$, $^{**}p<.01$, $^{***}p<.001$ (paired $t$-test, Bonferroni-corrected for 3 comparisons per model, $\alpha_{\text{adj}}=.017$).
\end{flushleft}
\end{table}

\paragraph{Planning vs.\ Reflection.}
We find that [planning/reflection/both/neither] provides the largest benefit. \textsc{Plan-Act} improves performance by [X\%/$d$=X.XX] on average, while \textsc{Act-Reflect} shows [smaller/larger/similar] gains of [X\%/$d$=X.XX].

\paragraph{Diminishing Returns.}
The \textsc{Full} architecture requires 3$\times$ the API calls but [does/does not] provide proportional improvement, suggesting [explanation].

\paragraph{Qualitative Observations.}
Examining generated plans, we observe:
\begin{itemize}
    \item [Observation about plan quality]
    \item [Observation about reflection quality]
    \item [Whether plans are actually followed]
\end{itemize}

\subsection{Cognitive Scaffolding Ablation}
\label{sec:exp-scaffolding}

Table~\ref{tab:scaffolding-results} shows the effect of providing computational tools.

\begin{table}[h]
\centering
\caption{Cognitive Scaffolding Comparison (Mean Profit $\pm$ SE)}
\label{tab:scaffolding-results}
\begin{tabular}{@{}lcccc@{}}
\toprule
\textbf{Model} & \textbf{None} & \textbf{Calculator} & \textbf{Math Prompt} & \textbf{Code} \\
\midrule
Claude Opus 4.5 & \$X.XX $\pm$ X.XX & \$X.XX $\pm$ X.XX & \$X.XX $\pm$ X.XX & \$X.XX $\pm$ X.XX \\
GPT-5.1 & \$X.XX $\pm$ X.XX & \$X.XX $\pm$ X.XX & \$X.XX $\pm$ X.XX & \$X.XX $\pm$ X.XX \\
Gemini 3 Pro & \$X.XX $\pm$ X.XX & \$X.XX $\pm$ X.XX & \$X.XX $\pm$ X.XX & \$X.XX $\pm$ X.XX \\
DeepSeek V3 & \$X.XX $\pm$ X.XX & \$X.XX $\pm$ X.XX & \$X.XX $\pm$ X.XX & \$X.XX $\pm$ X.XX \\
Llama 3.3 70B & \$X.XX $\pm$ X.XX & \$X.XX $\pm$ X.XX & \$X.XX $\pm$ X.XX & \$X.XX $\pm$ X.XX \\
\midrule
\textbf{Average} & \$X.XX & \$X.XX & \$X.XX & \$X.XX \\
\bottomrule
\end{tabular}
\begin{flushleft}
\small Significance vs.\ \textbf{None}: $^*p<.05$, $^{**}p<.01$, $^{***}p<.001$ (paired $t$-test, Bonferroni-corrected for 3 comparisons per model, $\alpha_{\text{adj}}=.017$).
\end{flushleft}
\end{table}

\paragraph{Tool Usage Patterns.}
When given the calculator tool, agents used it on [X\%] of turns. The code interpreter was used [more/less] frequently ([X\%] of turns). Common use cases included:
\begin{itemize}
    \item [Use case 1, e.g., profit margin calculation]
    \item [Use case 2, e.g., demand estimation]
    \item [Use case 3, e.g., price optimization]
\end{itemize}

\paragraph{Math Prompt vs.\ Calculator Tool.}
Interestingly, the math encouragement prompt [outperformed/matched/underperformed] the calculator tool ($p$ [</>] 0.05). This suggests that [interpretation about whether the bottleneck is computation vs.\ reasoning structure].

\paragraph{Calculation Accuracy.}
We audited 50 episodes with math prompts for calculation errors:
\begin{itemize}
    \item [X\%] of calculations were correct
    \item Common errors: [list error types]
    \item Errors [did/did not] correlate with poor decisions
\end{itemize}

\subsection{Failure Mode Analysis}
\label{sec:exp-failures}

We manually coded 100 randomly sampled episodes to identify systematic failure modes. Table~\ref{tab:failures} shows the prevalence of each failure type.

\begin{table}[h]
\centering
\caption{Failure Mode Prevalence (\% of episodes exhibiting behavior)}
\label{tab:failures}
\small
\begin{tabular}{@{}lcccccc@{}}
\toprule
\textbf{Failure Mode} & \textbf{Base} & \textbf{Aggr} & \textbf{Cons} & \textbf{Comp} & \textbf{Surv} & \textbf{Grow} \\
\midrule
Perishable overbuying & XX\% & XX\% & XX\% & XX\% & XX\% & XX\% \\
Ignoring weather forecast & XX\% & XX\% & XX\% & XX\% & XX\% & XX\% \\
Price anchoring (same price daily) & XX\% & XX\% & XX\% & XX\% & XX\% & XX\% \\
Stockout on peak days & XX\% & XX\% & XX\% & XX\% & XX\% & XX\% \\
Unnecessary location switching & XX\% & XX\% & XX\% & XX\% & XX\% & XX\% \\
Failure to buy cooler when beneficial & XX\% & XX\% & XX\% & XX\% & XX\% & XX\% \\
\bottomrule
\end{tabular}
\end{table}

\subsection{Episode Difficulty Analysis}
\label{sec:exp-difficulty}

Not all episodes are equally difficult. We analyze what makes certain weather sequences challenging and how difficulty correlates with agent performance.

\paragraph{Difficulty Metrics.}
We define episode difficulty along three dimensions:
\begin{itemize}
    \item \textbf{Weather volatility}: Standard deviation of demand multipliers across the 14-day sequence
    \item \textbf{Trap frequency}: Number of ``trap'' patterns (e.g., hot day followed by storm, incentivizing overbuying)
    \item \textbf{Forecast reliability}: How often tomorrow's forecast matches actual weather
\end{itemize}

\paragraph{Seed Difficulty Distribution.}
Figure~\ref{fig:difficulty} shows the distribution of Oracle-achievable profit across our 5 seeds, revealing variance in episode difficulty.

\begin{figure}[h]
\centering
\fbox{\parbox{0.8\textwidth}{\centering\vspace{2cm}\textbf{[PLACEHOLDER]}\\ Histogram of Oracle profit by seed, with seeds colored by difficulty tier (Easy/Medium/Hard). Annotate hardest and easiest seeds.\vspace{2cm}}}
\caption{Episode difficulty distribution across seeds. Seeds 1, 42, 100, 7, 2025 vary in achievable profit, with [X] seeds classified as ``hard'' (Oracle profit $<$ \$XX).}
\label{fig:difficulty}
\end{figure}

\paragraph{Performance Gap by Difficulty.}
Table~\ref{tab:difficulty-gap} shows how the human-LLM performance gap varies with episode difficulty.

\begin{table}[h]
\centering
\caption{Performance Gap by Episode Difficulty Tier}
\label{tab:difficulty-gap}
\begin{tabular}{@{}lccc@{}}
\toprule
\textbf{Difficulty} & \textbf{Human Profit} & \textbf{Best LLM Profit} & \textbf{Gap} \\
\midrule
Easy (top tercile) & \$X.XX $\pm$ X.XX & \$X.XX $\pm$ X.XX & \$X.XX \\
Medium & \$X.XX $\pm$ X.XX & \$X.XX $\pm$ X.XX & \$X.XX \\
Hard (bottom tercile) & \$X.XX $\pm$ X.XX & \$X.XX $\pm$ X.XX & \$X.XX \\
\bottomrule
\end{tabular}
\begin{flushleft}
\small Difficulty tiers based on Oracle profit terciles. Gap = Human $-$ LLM.
\end{flushleft}
\end{table}

We hypothesize that [humans/LLMs] will show larger performance gaps on hard episodes, suggesting [interpretation about robustness under adversity].

\paragraph{Trap Pattern Analysis.}
The most challenging pattern is the ``hot-storm trap'': a hot day (high demand) followed by a stormy day (near-zero demand). Agents who overbuy perishables on the hot day face spoilage. We identify [X] such traps across our seed set and find that [observation about agent behavior on these patterns].

\section{Results}
\label{sec:results}

We now present key findings organized by our pre-registered hypotheses.

\subsection{Hypothesis Testing}
\label{sec:hypothesis-results}

\begin{table}[h]
\centering
\caption{Pre-Registered Hypothesis Results}
\label{tab:hypotheses}
\small
\begin{tabular}{@{}cp{6.5cm}ccc@{}}
\toprule
\textbf{H\#} & \textbf{Hypothesis} & \textbf{Result} & \textbf{$p$-value} & \textbf{Effect} \\
\midrule
\multicolumn{5}{l}{\emph{Goal-Framing}} \\
H1 & \textsc{Competitive} increases risk-taking & \textcolor{gray}{TBD} & --- & $d$ = X.XX \\
H2 & \textsc{Conservative}/\textsc{Survival} reduces spoilage & \textcolor{gray}{TBD} & --- & $d$ = X.XX \\
H3 & \textsc{Aggressive} helps hot days, hurts bad days & \textcolor{gray}{TBD} & --- & $d$ = X.XX \\
H4 & \textsc{Growth} increases early exploration & \textcolor{gray}{TBD} & --- & $d$ = X.XX \\
H5 & Larger models more sensitive to framing & \textcolor{gray}{TBD} & --- & $r$ = X.XX \\
\midrule
\multicolumn{5}{l}{\emph{Architecture}} \\
H6 & \textsc{Plan-Act} reduces spoilage & \textcolor{gray}{TBD} & --- & $d$ = X.XX \\
H7 & \textsc{Act-Reflect} improves in late game & \textcolor{gray}{TBD} & --- & $d$ = X.XX \\
H8 & \textsc{Full} shows diminishing returns & \textcolor{gray}{TBD} & --- & $d$ = X.XX \\
H9 & Plans frequently abandoned mid-episode & \textcolor{gray}{TBD} & --- & X\% \\
\midrule
\multicolumn{5}{l}{\emph{Scaffolding}} \\
H10 & Calculator reduces pricing errors & \textcolor{gray}{TBD} & --- & $d$ = X.XX \\
H11 & Math prompts help even without tools & \textcolor{gray}{TBD} & --- & $d$ = X.XX \\
H12 & Code interpreter underutilized & \textcolor{gray}{TBD} & --- & X\% use \\
H13 & All LLMs underperform Reactive on spoilage & \textcolor{gray}{TBD} & --- & --- \\
\bottomrule
\end{tabular}
\begin{flushleft}
\small Effect sizes: Cohen's $d$ for continuous outcomes (small $\geq 0.2$, medium $\geq 0.5$, large $\geq 0.8$); Pearson's $r$ for correlations. All tests use Bonferroni correction within each hypothesis family.
\end{flushleft}
\end{table}

\subsection{Key Finding 1: [Title TBD]}
\label{sec:finding1}

% Placeholder for main finding
\emph{[Describe the most surprising or important finding here. Include statistical details and interpretation.]}

\subsection{Key Finding 2: [Title TBD]}
\label{sec:finding2}

% Placeholder for second finding
\emph{[Describe another key finding, potentially about model differences or a specific condition effect.]}

\subsection{Key Finding 3: [Title TBD]}
\label{sec:finding3}

% Placeholder for third finding
\emph{[Describe a finding about failure modes or qualitative behavioral patterns.]}

\subsection{Model Comparison}
\label{sec:model-comparison}

Figure~\ref{fig:model-radar} provides a multi-dimensional comparison of model capabilities.

\begin{figure}[h]
\centering
\fbox{\parbox{0.8\textwidth}{\centering\vspace{2cm}\textbf{[PLACEHOLDER]}\\ Radar/spider chart showing each model's performance on:\\ - Total profit (normalized)\\ - Spoilage rate (inverted)\\ - Weather adaptation\\ - Recovery score\\ - Prompt sensitivity\vspace{2cm}}}
\caption{Multi-dimensional model comparison. [Model X] shows the best overall profile; [Model Y] excels at [specific dimension] but struggles with [other dimension].}
\label{fig:model-radar}
\end{figure}

\subsection{Comparison with Reinforcement Learning Baseline}
\label{sec:rl-baseline}

To contextualize LLM agent performance, we trained a traditional reinforcement learning agent using Proximal Policy Optimization (PPO)~\citep{schulman2017proximal} via Stable Baselines3~\citep{stable-baselines3}. This provides a reference point for what a learning-based approach can achieve with full access to the environment dynamics.

\paragraph{Architecture and Training.}
The PPO agent uses a two-layer MLP policy with 64 hidden units per layer ($\sim$12K parameters), processing a 27-dimensional continuous observation encoding weather conditions, inventory levels, cash, reputation, and temporal features. Actions are decoded from an 8-dimensional continuous output space covering price setting and inventory purchases. The agent was trained for 4 million timesteps using:
\begin{itemize}[noitemsep]
    \item Generalized Advantage Estimation (GAE) with $\lambda = 0.95$
    \item Rollout buffer of 4,096 steps across 8 parallel environments
    \item Linearly decaying learning rate from $3 \times 10^{-4}$ to 0
    \item Observation normalization via running statistics
    \item \textbf{Randomized episode seeds} for generalization across weather patterns
\end{itemize}

Training completed in approximately 12 minutes on an AMD Ryzen 7 9800X3D CPU at $\sim$5,500 steps/second. Notably, the small MLP policy does not benefit from GPU acceleration---the environment simulation (running in Python) is the primary bottleneck.

\paragraph{Results.}
Table~\ref{tab:rl-baseline} compares the PPO agent against LLM agents on the same 10 evaluation seeds used throughout this paper.

\begin{table}[h]
\centering
\caption{PPO Baseline vs.\ LLM Agents on Paper Methodology Seeds}
\label{tab:rl-baseline}
\small
\begin{tabular}{@{}lccccc@{}}
\toprule
\textbf{Agent} & \textbf{Mean Profit} & \textbf{Std Dev} & \textbf{Min} & \textbf{Max} & \textbf{Cups Sold} \\
\midrule
PPO Baseline (4M steps) & \$584.45 & \$98.31 & \$475.88 & \$778.87 & 573.6 \\
\midrule
\emph{[Best LLM Agent]} & \emph{[TBD]} & \emph{[TBD]} & \emph{[TBD]} & \emph{[TBD]} & \emph{[TBD]} \\
\emph{[Median LLM Agent]} & \emph{[TBD]} & \emph{[TBD]} & \emph{[TBD]} & \emph{[TBD]} & \emph{[TBD]} \\
Random Baseline & \emph{[TBD]} & \emph{[TBD]} & \emph{[TBD]} & \emph{[TBD]} & \emph{[TBD]} \\
\bottomrule
\end{tabular}
\end{table}

\paragraph{Interpretation.}
The PPO agent achieves strong performance (\$584 mean profit) through millions of trial-and-error interactions with the environment, learning implicit patterns about weather-dependent pricing and inventory management. However, this comes with important caveats:

\begin{enumerate}[noitemsep]
    \item \textbf{No interpretability}: Unlike LLM agents that produce natural language reasoning, PPO's policy is a black-box neural network with no explanatory capacity.
    
    \item \textbf{No transfer}: The PPO agent is trained specifically on this environment; LLM agents apply general reasoning that could transfer to similar domains.
    
    \item \textbf{Sample efficiency}: PPO required 4M environment interactions ($\sim$285K episodes). LLM agents operate zero-shot or with minimal in-context examples.
    
    \item \textbf{High variance}: The PPO agent shows $\pm$\$98 standard deviation across seeds, reflecting genuine difficulty variance in the environment rather than reasoning inconsistency.
\end{enumerate}

The PPO baseline establishes that LemonadeBench is learnable through gradient-based optimization, while highlighting the distinct value proposition of LLM agents: interpretable reasoning, transfer potential, and sample efficiency---at the cost of [higher/lower/comparable] absolute performance.


\section{Discussion}
\label{sec:discussion}

\subsection{Implications for Agent Deployment}
\label{sec:implications}

Our findings have practical implications for deploying LLM agents in decision-making contexts:

\begin{enumerate}
    \item \textbf{Prompt framing matters}: The same model can exhibit meaningfully different behavior based on goal framing. Practitioners should carefully consider how objectives are communicated to agents.
    
    \item \textbf{Risk calibration is malleable}: \textsc{Competitive} and \textsc{Aggressive} framings [do/do not] reliably increase risk-taking, suggesting that [LLMs can/cannot] be steered toward desired risk profiles through prompting alone.
    
    \item \textbf{Architecture choice depends on task}: [Planning/Reflection/Neither] provided the largest benefit in our setting. For tasks requiring multi-day coordination, [recommendation]. For tasks requiring adaptation to feedback, [recommendation].
    
    \item \textbf{Tools vs.\ prompts}: The [calculator tool/math prompt] provided [larger/similar/smaller] benefits, suggesting that [structured reasoning / reliable computation] is the bottleneck for LLM decision-making. Practitioners may not need complex tool infrastructure if [finding about prompts].
    
    \item \textbf{Inventory management remains challenging}: Even with explicit forecasts and calculation tools provided, all models struggled with perishable inventory---a cautionary finding for supply chain applications.
\end{enumerate}

\subsection{When Do Agent Architectures Help?}
\label{sec:arch-discussion}

Our architecture ablation provides nuanced insights into when additional cognitive phases benefit agent performance.

\paragraph{Planning Benefits.}
\textsc{Plan-Act} [improved/did not improve] inventory management, supporting/contradicting the intuition that explicit planning helps with multi-day resource allocation. Interestingly, we observed that [observation about plan quality or adherence].

\paragraph{Reflection Benefits.}
\textsc{Act-Reflect} showed [stronger/weaker] effects in the second half of episodes, suggesting that reflection [does/does not] enable within-episode learning. The quality of reflections [varied/was consistent] across models, with [observation about which models reflected more effectively].

\paragraph{Cost-Benefit Tradeoff.}
\textsc{Full} required approximately 3$\times$ the API calls of \textsc{React}, at roughly [\$X.XX/\$X.XX] per episode. The marginal benefit over \textsc{Plan-Act} or \textsc{Act-Reflect} was [$d$ = X.XX], suggesting [recommendation about when full architecture is warranted].

\subsection{The Role of Cognitive Scaffolding}
\label{sec:scaffolding-discussion}

Our scaffolding experiments reveal insights about the nature of LLM reasoning limitations in economic tasks.

\paragraph{Computation vs.\ Reasoning.}
If the bottleneck were arithmetic errors, calculator access should have helped. If the bottleneck were lack of structured thinking, math prompts should have helped. We found [which helped more], suggesting the primary limitation is [computation/reasoning structure/something else].

\paragraph{Underutilization of Tools.}
Agents used the code interpreter on only [X\%] of opportunities. When they did use it, the code was [sophisticated/trivial], suggesting that [interpretation about agent tool-use capabilities].

\paragraph{Emergent Behaviors.}
We observed several unexpected patterns with scaffolding:
\begin{itemize}
    \item [Unexpected behavior 1]
    \item [Unexpected behavior 2]
\end{itemize}

\subsection{Connections to Behavioral Economics}
\label{sec:behavioral}

Our goal-framing conditions were inspired by well-studied phenomena in human behavioral economics:

\begin{itemize}
    \item \textbf{Loss aversion} \citep{kahneman1979prospect}: \textsc{Survival} framing explicitly emphasizes losses. We [did/did not] observe increased conservatism, suggesting LLMs [do/do not] exhibit loss-averse behavior.
    
    \item \textbf{Tournament effects}: Economics literature shows humans take more risks in competitive settings. Our \textsc{Competitive} condition [replicates/fails to replicate] this effect in LLMs.
    
    \item \textbf{Temporal discounting} \citep{frederick2002time}: \textsc{Growth} framing encourages long-term thinking. Agents [did/did not] sacrifice early profits for reputation building, indicating [presence/absence] of temporal reasoning.
\end{itemize}

\subsection{Limitations}
\label{sec:limitations}

Several limitations constrain the generalizability of our findings:

\begin{enumerate}
    \item \textbf{Single domain}: While the lemonade stand is interpretable, it may not generalize to other business domains with different dynamics (e.g., manufacturing, services).

    \item \textbf{Prompt sensitivity}: Our specific phrasings of goal-framing conditions may not capture the full space of possible framings. Different wording with equivalent semantic content may yield different behavioral responses.

    \item \textbf{No fine-tuning}: We evaluated base models without task-specific training; fine-tuned agents may show different patterns.

    \item \textbf{API model versions}: Model behavior may change with provider updates; we report exact model versions and evaluation date in Appendix~\ref{app:versions}.

    \item \textbf{Sampling determinism}: We use $T=0$ for reproducibility, which eliminates exploration of the model's uncertainty. Higher temperatures might reveal different strategic behaviors or robustness patterns, though at the cost of reproducibility.
\end{enumerate}

\subsection{Future Work}
\label{sec:future}

This work opens several avenues for future research:

\begin{itemize}
    \item \textbf{Multi-agent settings}: How do framing effects change when multiple agents compete directly in the same environment?
    
    \item \textbf{Learning from feedback}: Can agents improve their strategies through explicit feedback or self-reflection mechanisms?
    
    \item \textbf{Hybrid approaches}: Combining LLM reasoning with traditional optimization (e.g., using LLMs for strategy selection and solvers for execution).
    
    \item \textbf{Transfer to real domains}: Validating whether insights transfer to real business decision-making tasks.
\end{itemize}

\section{Conclusion}
\label{sec:conclusion}

We introduced LemonadeBench, a benchmark for evaluating LLM agents on multi-day sequential business decision-making. Our key contributions include:

\begin{enumerate}
    \item A novel, interpretable environment requiring sustained strategic reasoning across 14 days with compounding decisions, perishable inventory, and stochastic weather.
    
    \item The first systematic study of goal-framing effects on LLM economic behavior, testing six conditions inspired by behavioral economics.
    
    \item Empirical evaluation of six frontier models revealing [key finding about model performance].
    
    \item Evidence that [key finding about framing effects], with implications for agent deployment and prompt engineering.
\end{enumerate}

Our results suggest that while current LLMs can engage in business reasoning, they [struggle with / show promise in] maintaining coherent long-horizon strategies. The sensitivity of agent behavior to goal framing raises important questions about the reliability of LLM agents in high-stakes decision contexts---and opportunities for steering agent behavior through careful prompt design.

LemonadeBench is publicly available at \url{https://github.com/Shaun3141/LemonadeBench}, including the environment, evaluation harness, all experimental data, and an interactive web client for human comparison.

\section*{Broader Impact Statement}

This work studies how LLM agents make economic decisions and how prompt framing influences their behavior. We identify both positive impacts and potential risks.

\paragraph{Positive Impacts.}
\begin{itemize}
    \item \textbf{Safer agent deployment}: Understanding agent economic biases \emph{before} deployment in real-world systems (e.g., automated trading, supply chain management) can prevent costly failures.
    \item \textbf{Alignment research}: Our findings on goal-framing effects contribute to understanding how LLMs interpret and act on objectives---a core challenge in AI alignment.
    \item \textbf{Educational value}: The interpretable lemonade stand domain provides an accessible testbed for teaching agent evaluation and behavioral economics concepts.
\end{itemize}

\paragraph{Potential Risks.}
\begin{itemize}
    \item \textbf{Manipulation of agent systems}: Knowledge of how goal framing affects agent behavior could be exploited to manipulate LLM-based systems (e.g., crafting prompts that induce excessive risk-taking in trading agents).
    \item \textbf{Overconfidence in benchmarks}: Strong performance on LemonadeBench does not guarantee safe behavior in more complex real-world economic environments.
    \item \textbf{Anthropomorphization}: Describing LLM behavior in terms of ``risk aversion'' or ``competitive drive'' may incorrectly suggest human-like cognition.
\end{itemize}

\paragraph{Mitigations.}
Our benchmark is explicitly designed as a low-stakes simulation with no real-world financial consequences. We emphasize throughout the paper that findings may not transfer to production systems without further validation. We release all code and data to enable scrutiny and reproducibility.

\section*{Acknowledgments}

% TODO: Acknowledgments



\bibliographystyle{plainnat}
\bibliography{references}

% Appendices
\appendix
\section{Environment Details}
\label{app:environment}

\subsection{Supply Costs and Recipes}

Table~\ref{tab:supplies} details the supply economics. Bulk purchasing provides 10\% discount at medium quantities and 20\% at large quantities.

\begin{table}[h]
\centering
\caption{Supply Costs and Usage}
\label{tab:supplies}
\begin{tabular}{@{}lcccc@{}}
\toprule
\textbf{Supply} & \textbf{Base Cost} & \textbf{Per Cup} & \textbf{Cups/Unit} & \textbf{Shelf Life} \\
\midrule
Lemons & \$0.25 & 0.25 & 4 & 3 days \\
Sugar (bag) & \$0.50 & 0.10 & 10 & $\infty$ \\
Cups & \$0.05 & 1.00 & 1 & $\infty$ \\
Ice (bag) & \$0.25 & 0.20 & 5 & 1 day* \\
\bottomrule
\end{tabular}
\begin{flushleft}
\small *Ice melts 100\% overnight without cooler, 50\% with cooler upgrade (\$2.50).
\end{flushleft}
\end{table}

\subsection{Bulk Pricing Tiers}

\begin{itemize}
    \item \textbf{Lemons}: Single (0\%), Dozen/12+ (10\% off), Crate/144+ (20\% off)
    \item \textbf{Sugar}: Single (0\%), Case/5+ (10\% off), Pallet/20+ (20\% off)
    \item \textbf{Cups}: Pack/10 (0\%), Sleeve/50+ (10\% off), Case/250+ (20\% off)
    \item \textbf{Ice}: Single (0\%), Cooler/5+ (10\% off), Delivery/20+ (20\% off)
\end{itemize}

\subsection{Default Configuration}

\begin{itemize}
    \item Season length: 14 days
    \item Starting cash: \$20.00
    \item Starting inventory: 10 lemons, 5 sugar bags, 50 cups, 5 ice bags
    \item Base customers: 50/day (before modifiers)
    \item Optimal price: \$0.50 (95\% conversion)
    \item Maximum tolerated price: \$2.00 (near-zero conversion)
\end{itemize}


\section{Prompt Templates}
\label{app:prompts}

We use structured tool calling with Claude's native \texttt{tool\_use} feature. The system prompt and observation format are shown below.

\subsection{System Prompt}

\begin{verbatim}
You are an AI agent running a lemonade stand business. 
Your goal is to maximize profit over a 14-day summer season.

## Game Mechanics

**Weather Effects on Demand:**
- HOT: Very high demand, customers willing to pay premium prices
- SUNNY: High demand, good day for sales
- CLOUDY: Moderate demand
- RAINY: Low demand, few customers venture out
- STORMY: Very low demand

**Pricing Strategy:**
- Higher prices = fewer customers but more profit per sale
- Lower prices = more customers but less profit per sale
- The "sweet spot" is usually around $0.50-$1.00 depending on weather

**Inventory:**
- Lemons: $0.25 each, expire after 3 days
- Sugar bags: $0.50 each, don't expire
- Cups: $0.05 each, don't expire
- Ice: $0.25/bag, melts overnight unless you have a cooler

**Recipe per cup:**
- 0.25 lemons (4 cups per lemon)
- 0.1 sugar bags (10 cups per bag)
- 0.2 ice bags (5 cups per bag) - optional but boosts hot day sales

## Strategy Tips
1. Watch the weather forecast to plan inventory
2. Don't overbuy perishables (lemons, ice)
3. Higher prices on hot/sunny days, lower on bad weather
4. Build reputation by serving customers well
5. Advertising helps on good weather days
\end{verbatim}

\subsection{Observation Format}

Each turn, the agent receives a structured observation:

\begin{verbatim}
# Day 3 Results

## Current Conditions
- Weather: HOT (98°F)
- Tomorrow's Forecast: sunny
- Days Remaining: 11

## Finances
- Cash: $24.50
- Total Profit So Far: $4.50

## Yesterday's Results
- Cups Sold: 45
- Revenue: $33.75
- Costs: $8.25
- Daily Profit: $25.50
- Customers Served: 45
- Customers Turned Away: 12

## Current Inventory
- Lemons: 8 (expiring tomorrow: 3)
- Sugar Bags: 4.5
- Cups: 42
- Ice Bags: 0

## Stand Status
- Location: park
- Owned Upgrades: None
- Reputation: 0.58
\end{verbatim}

\subsection{Action Tool Schema}

\begin{verbatim}
{
  "name": "take_action",
  "input_schema": {
    "type": "object",
    "properties": {
      "reasoning": {
        "type": "string",
        "description": "Brief explanation of your strategy"
      },
      "price_per_cup": {
        "type": "integer",
        "description": "Price in cents (e.g., 75 = $0.75)"
      },
      "buy_lemons": {"type": "integer", "default": 0},
      "buy_sugar": {"type": "integer", "default": 0},
      "buy_cups": {"type": "integer", "default": 0},
      "buy_ice": {"type": "integer", "default": 0},
      "advertising_spend": {"type": "integer", "default": 0},
      "buy_upgrade": {
        "type": "string",
        "enum": ["cooler", null]
      },
      "location": {
        "type": "string",
        "enum": ["park", "downtown", "mall", "pool", null]
      }
    },
    "required": ["reasoning", "price_per_cup"]
  }
}
\end{verbatim}

The \texttt{reasoning} field captures agent chain-of-thought, enabling qualitative analysis of decision-making patterns. All runs log the full conversation history, actions, and outcomes for post-hoc analysis.


\section{Model Versions}
\label{app:versions}

All experiments were conducted in December 2025 using OpenRouter as a unified API gateway to access models from multiple providers. This approach ensures consistent API behavior and enables reproducible comparisons across providers.

\subsection{Model Selection Criteria}

We selected 20 models across five tiers based on:
\begin{enumerate}
    \item \textbf{Agentic capability}: Demonstrated tool/function calling support
    \item \textbf{Reasoning benchmarks}: Performance on MATH-500, ARC-AGI, SWE-Bench
    \item \textbf{Cost diversity}: Range from \$0.075/M to \$75/M tokens
    \item \textbf{Accessibility}: Including open-source options for reproducibility
    \item \textbf{Provider diversity}: Representation from 7 providers
\end{enumerate}

\subsection{Evaluated Models}

\begin{table}[h]
\centering
\caption{Complete Model Specifications}
\label{tab:versions}
\small
\begin{tabular}{@{}llrrl@{}}
\toprule
\textbf{Model} & \textbf{Provider} & \textbf{Input/M} & \textbf{Output/M} & \textbf{Tier} \\
\midrule
\multicolumn{5}{l}{\emph{Tier 1: Premium Reasoning}} \\
claude-sonnet-4 & Anthropic & \$3.00 & \$15.00 & Premium \\
claude-opus-4.5 & Anthropic & \$15.00 & \$75.00 & Premium \\
o1 & OpenAI & \$15.00 & \$60.00 & Premium \\
gpt-5.1 & OpenAI & \$5.00 & \$15.00 & Premium \\
gemini-3-pro & Google & \$2.50 & \$15.00 & Premium \\
\midrule
\multicolumn{5}{l}{\emph{Tier 2: Balanced Performance}} \\
claude-3.5-sonnet & Anthropic & \$3.00 & \$15.00 & Balanced \\
o3-mini & OpenAI & \$1.10 & \$4.40 & Balanced \\
gemini-2.5-flash & Google & \$0.30 & \$2.50 & Balanced \\
grok-2 & X.AI & \$5.00 & \$10.00 & Balanced \\
\midrule
\multicolumn{5}{l}{\emph{Tier 3: Cost-Effective}} \\
deepseek-r1 & DeepSeek & \$0.55 & \$2.19 & Value \\
deepseek-v3 & DeepSeek & \$0.27 & \$1.10 & Value \\
qwq-32b & Qwen & \$0.20 & \$0.20 & Value \\
mistral-small-3 & Mistral & \$0.10 & \$0.30 & Value \\
\midrule
\multicolumn{5}{l}{\emph{Tier 4: Open Source}} \\
llama-3.3-70b-instruct & Meta & \$0.40 & \$0.40 & Open \\
llama-3.1-405b-instruct & Meta & \$2.00 & \$2.00 & Open \\
qwen-2.5-72b-instruct & Qwen & \$0.35 & \$0.40 & Open \\
mistral-large & Mistral & \$2.00 & \$6.00 & Open \\
\midrule
\multicolumn{5}{l}{\emph{Tier 5: Fast \& Efficient}} \\
claude-3.5-haiku & Anthropic & \$0.80 & \$4.00 & Fast \\
gpt-4o-mini & OpenAI & \$0.15 & \$0.60 & Fast \\
gemini-flash-1.5 & Google & \$0.075 & \$0.30 & Fast \\
\bottomrule
\end{tabular}
\begin{flushleft}
\small Prices in USD per million tokens via OpenRouter (December 2025).
\end{flushleft}
\end{table}

\subsection{Model Capabilities}

Table~\ref{tab:model-capabilities} summarizes key capabilities relevant to the LemonadeBench task.

\begin{table}[h]
\centering
\caption{Model Capabilities for Agentic Tasks}
\label{tab:model-capabilities}
\small
\begin{tabular}{@{}lcccr@{}}
\toprule
\textbf{Model} & \textbf{Tool Calling} & \textbf{Context} & \textbf{Reasoning} & \textbf{MATH-500} \\
\midrule
claude-opus-4.5 & \checkmark & 200K & Extended & --- \\
claude-sonnet-4 & \checkmark & 200K & Standard & --- \\
o1 & \checkmark & 200K & CoT & 94.8\% \\
o3-mini & \checkmark & 200K & CoT & --- \\
gpt-5.1 & \checkmark & 200K & Extended & --- \\
gemini-3-pro & \checkmark & 2M & Extended & --- \\
deepseek-r1 & \checkmark & 64K & CoT & 97.3\% \\
deepseek-v3 & \checkmark & 128K & Standard & 92.5\% \\
qwq-32b & \checkmark & 32K & CoT & --- \\
llama-3.3-70b & \checkmark & 128K & Standard & --- \\
llama-3.1-405b & \checkmark & 128K & Standard & --- \\
qwen-2.5-72b & \checkmark & 128K & Standard & --- \\
\bottomrule
\end{tabular}
\begin{flushleft}
\small CoT = Chain-of-Thought reasoning. MATH-500 scores from model release papers where available.
\end{flushleft}
\end{table}

\subsection{API Configuration}

All models were accessed via OpenRouter's unified API with the following settings:
\begin{itemize}
    \item \textbf{Base URL}: \texttt{https://openrouter.ai/api/v1}
    \item \textbf{Max tokens}: 1024 per response
    \item \textbf{Temperature}: 0 (greedy decoding)
    \item \textbf{Top-p}: 1.0
    \item \textbf{Tool choice}: Required (forced function calling)
\end{itemize}

Greedy decoding ($T=0$) ensures full reproducibility and isolates the effects of goal-framing prompts from sampling stochasticity. Given the same environment seed and goal-framing condition, a model produces identical behavior across runs. Environmental variance across the 5--10 random seeds provides sufficient stochastic variation for statistical analysis, making sampling variance unnecessary and scientifically undesirable.

\subsection{Cost Analysis}

Total evaluation cost across 1,000 episodes reflects three experimental components:

\begin{table}[h]
\centering
\caption{Estimated Evaluation Cost by Experiment Type}
\label{tab:cost}
\begin{tabular}{@{}lrrr@{}}
\toprule
\textbf{Tier} & \textbf{Goal-Framing} & \textbf{Ablations} & \textbf{Total Est. Cost} \\
 & \textbf{(30 eps)} & \textbf{(80 eps)} & \\
\midrule
Premium (5 models) & \$36--60 & \$96--160 & \$132--220 \\
Balanced (4 models) & \$24--48 & --- & \$24--48 \\
Value (4 models) & \$6--12 & \$6--12 & \$12--24 \\
Open Source (4 models) & \$12--24 & \$12--24 & \$24--48 \\
Fast (3 models) & \$3--6 & --- & \$3--6 \\
\midrule
\multicolumn{3}{r}{\textbf{Grand Total (1,000 episodes):}} & \textbf{\$195--346} \\
\bottomrule
\end{tabular}
\begin{flushleft}
\small Estimates based on ~2,000 input tokens and ~500 output tokens per turn, 14 turns per episode. Ablations include 5 flagship models only (see Table~\ref{tab:model-categories}). Architecture ablations with \textsc{Full} loop require 3$\times$ the base cost due to planning and reflection phases.
\end{flushleft}
\end{table}

\paragraph{Cost-Effectiveness.}
At the reported price ranges, evaluating all 20 models across goal-framing conditions costs approximately \$81--150 (600 episodes). The most cost-effective tier for comprehensive evaluation is Value (DeepSeek R1, DeepSeek V3, QwQ-32B), providing strong reasoning at <\$0.50/episode.

\subsection{Reproducibility}

To reproduce our experiments:

\begin{enumerate}
    \item Obtain an OpenRouter API key at \texttt{https://openrouter.ai/keys}
    \item Set environment variable: \texttt{export OPENROUTER\_API\_KEY="your-key"}
    \item Run the batch configuration:
    \begin{verbatim}
    lemonade batch examples/batch_config.yaml --parallel 4
    \end{verbatim}
\end{enumerate}

The exact configuration file is included in the supplementary materials and available at the project repository.

\section{Full Goal-Framing Prompts}
\label{app:full-prompts}

For reproducibility, we provide the complete text of each goal-framing condition as prepended to the base system prompt.

\subsection{\textsc{Aggressive}}
\begin{quote}
\small
``You are an aggressive entrepreneur who takes calculated risks to maximize returns. Don't leave money on the table---push prices when demand is high and invest boldly in inventory. Fortune favors the bold. Your competitors are playing it safe; you're here to win big.''
\end{quote}

\subsection{\textsc{Conservative}}
\begin{quote}
\small
``You are a cautious business owner who prioritizes avoiding losses over maximizing gains. Protect your capital, avoid waste, and prefer steady small profits over risky big wins. A bird in hand is worth two in the bush. Consistency and reliability will serve you better than gambling on big days.''
\end{quote}

\subsection{\textsc{Competitive}}
\begin{quote}
\small
``You are competing against 10 other lemonade stands in a tournament. Only the top 3 profit earners will win prizes. The average stand makes about \$25-30 profit over the season. You need to significantly outperform this average to place. Second place is first loser---aim for the top.''
\end{quote}

\subsection{\textsc{Survival}}
\begin{quote}
\small
``Your family depends on this business. You started with \$20, and if you end the season with less than \$20, you will have failed your family. Survival is the absolute priority. Do not take unnecessary risks. Losing money is not an option. Every decision should prioritize capital preservation.''
\end{quote}

\subsection{\textsc{Growth}}
\begin{quote}
\small
``You're building a lemonade empire. This 14-day season is just the beginning of a much longer journey. Focus on building strong reputation and deeply learning the market dynamics, even if it costs some short-term profit. Experiment with locations, understand customer behavior, and invest in your stand's capabilities. The lessons you learn now will pay dividends for years to come.''
\end{quote}


\section{Agent Loop Implementations}
\label{app:loops}

We provide detailed pseudocode for each agent architecture to ensure reproducibility.

\subsection{\textsc{React} (Baseline)}

\begin{verbatim}
def run_episode(env, llm):
    obs = env.reset()
    messages = [format_observation(obs)]
    
    while not obs.done:
        response = llm.generate(messages, tools=[ACTION_TOOL])
        action = parse_action(response)
        obs = env.step(action)
        messages.append(response)
        messages.append(format_result(obs))
    
    return obs.total_profit
\end{verbatim}

\subsection{\textsc{Plan-Act}}

\begin{verbatim}
PLANNING_PROMPT = """
Before taking action, create a plan for the next 3-5 days:
1. Weather outlook and expected demand
2. Inventory needs and purchase schedule
3. Pricing strategy by weather condition
4. Location considerations
5. Key risks and contingencies

Output your plan, then take today's action.
"""

def run_episode(env, llm):
    obs = env.reset()
    messages = [format_observation(obs)]
    
    while not obs.done:
        # Planning phase
        plan_prompt = PLANNING_PROMPT + format_observation(obs)
        plan_response = llm.generate(messages + [plan_prompt])
        
        # Action phase (with plan context)
        action_prompt = f"Given your plan:\n{plan_response}\n\nNow take action:"
        response = llm.generate(
            messages + [plan_response, action_prompt], 
            tools=[ACTION_TOOL]
        )
        action = parse_action(response)
        obs = env.step(action)
        
        messages.extend([plan_response, response, format_result(obs)])
    
    return obs.total_profit
\end{verbatim}

\subsection{\textsc{Act-Reflect}}

\begin{verbatim}
REFLECTION_PROMPT = """
Reflect on yesterday's results:
1. What worked well?
2. What could you have done better?
3. What surprised you?
4. What will you do differently?

Then decide today's action.
"""

def run_episode(env, llm):
    obs = env.reset()
    messages = [format_observation(obs)]
    
    while not obs.done:
        # Action phase
        response = llm.generate(messages, tools=[ACTION_TOOL])
        action = parse_action(response)
        obs = env.step(action)
        messages.append(response)
        
        if not obs.done:
            # Reflection phase
            reflect_prompt = format_result(obs) + REFLECTION_PROMPT
            reflection = llm.generate(messages + [reflect_prompt])
            messages.extend([reflect_prompt, reflection])
        
        messages.append(format_observation(obs))
    
    return obs.total_profit
\end{verbatim}

\subsection{\textsc{Full} (Plan + Reflect)}

\begin{verbatim}
def run_episode(env, llm):
    obs = env.reset()
    messages = [format_observation(obs)]
    current_plan = None
    
    while not obs.done:
        # Planning phase (every turn or when plan invalidated)
        plan_prompt = PLANNING_PROMPT + format_observation(obs)
        if current_plan:
            plan_prompt += f"\nPrevious plan: {current_plan}\nUpdate if needed."
        current_plan = llm.generate(messages + [plan_prompt])
        
        # Action phase
        action_prompt = f"Execute your plan:\n{current_plan}"
        response = llm.generate(
            messages + [current_plan, action_prompt],
            tools=[ACTION_TOOL]
        )
        action = parse_action(response)
        obs = env.step(action)
        
        if not obs.done:
            # Reflection phase
            reflect_prompt = format_result(obs) + REFLECTION_PROMPT
            reflect_prompt += f"\nDoes your plan still make sense? Plan: {current_plan}"
            reflection = llm.generate(messages + [reflect_prompt])
            messages.extend([current_plan, response, reflect_prompt, reflection])
        
        messages.append(format_observation(obs))
    
    return obs.total_profit
\end{verbatim}

\subsection{Tool Definitions}

\subsubsection{Calculator Tool}
\begin{verbatim}
{
  "name": "calculate",
  "description": "Evaluate a mathematical expression. Use for profit 
                  calculations, demand estimates, break-even analysis.",
  "input_schema": {
    "type": "object",
    "properties": {
      "expression": {
        "type": "string",
        "description": "Math expression, e.g., '(50 * 0.75) - (12 * 0.25)'"
      }
    },
    "required": ["expression"]
  }
}
\end{verbatim}

\subsubsection{Code Interpreter Tool}
\begin{verbatim}
{
  "name": "run_python",
  "description": "Execute Python code. Use for complex calculations,
                  optimization, or data analysis.",
  "input_schema": {
    "type": "object",
    "properties": {
      "code": {
        "type": "string",
        "description": "Python code to execute. Has access to numpy, scipy."
      }
    },
    "required": ["code"]
  }
}
\end{verbatim}


\section{Sample Agent Reasoning}
\label{app:reasoning}

Below are representative examples of agent reasoning across different conditions and models.

\subsection{Example: Weather Adaptation}
\textbf{Condition}: \textsc{Baseline} | \textbf{Model}: Claude 3.5 Sonnet | \textbf{Day}: 5 | \textbf{Weather}: Hot (98°F)

\begin{quote}
\small\ttfamily
[PLACEHOLDER: Insert actual agent reasoning from a hot day showing price adjustment]
\end{quote}

\subsection{Example: Risk-Taking Under Competition}
\textbf{Condition}: \textsc{Competitive} | \textbf{Model}: GPT-5.1 | \textbf{Day}: 10 | \textbf{Weather}: Cloudy

\begin{quote}
\small\ttfamily
[PLACEHOLDER: Insert actual agent reasoning showing competitive risk behavior]
\end{quote}

\subsection{Example: Conservative Inventory Management}
\textbf{Condition}: \textsc{Survival} | \textbf{Model}: Gemini 3 Pro | \textbf{Day}: 3 | \textbf{Weather}: Rainy

\begin{quote}
\small\ttfamily
[PLACEHOLDER: Insert actual agent reasoning showing conservative purchasing]
\end{quote}

\section{Reproducibility Checklist}
\label{app:reproducibility}

We provide detailed information to enable full reproducibility of our experiments.

\subsection{Code and Data Availability}

\begin{itemize}
    \item \textbf{Repository}: \url{https://github.com/Shaun3141/LemonadeBench}
    \item \textbf{License}: BSD-3-Clause (permissive open source)
    \item \textbf{Environment code}: \texttt{lemonade\_bench/server/lemonade\_environment.py}
    \item \textbf{Agent implementations}: \texttt{lemonade\_bench/agents/}
    \item \textbf{Evaluation harness}: \texttt{lemonade\_bench/harness/}
    \item \textbf{Raw experimental data}: Available in repository under \texttt{runs/}
    \item \textbf{Analysis notebooks}: Jupyter notebooks for reproducing all figures and tables
\end{itemize}

\subsection{Compute Requirements}

\begin{itemize}
    \item \textbf{Environment}: Runs on any machine with Python 3.11+; no GPU required
    \item \textbf{Episode runtime}: $<$1 second per episode (environment only); 30--120 seconds with LLM API calls
    \item \textbf{Total API cost}: Approximately \$[XXX] for all 960 episodes
    \item \textbf{Human baseline collection}: Approximately \$[XXX] for 60 human episodes (20 participants $\times$ 3 episodes)
\end{itemize}

\subsection{Hyperparameters}

\begin{table}[h]
\centering
\caption{Hyperparameters Used Across All Experiments}
\label{tab:hyperparams}
\begin{tabular}{@{}ll@{}}
\toprule
\textbf{Parameter} & \textbf{Value} \\
\midrule
\multicolumn{2}{l}{\emph{Environment}} \\
Season length & 14 days \\
Starting cash & \$20.00 (2000 cents) \\
Random seeds (LLM) & 1--20 \\
Random seeds (Human) & 21--40 \\
\midrule
\multicolumn{2}{l}{\emph{LLM Settings}} \\
Temperature & 0.7 \\
Max tokens & 1024 \\
Tool choice & Forced (\texttt{take\_action}) \\
Retry on error & Up to 3 attempts \\
\midrule
\multicolumn{2}{l}{\emph{Evaluation}} \\
Episodes per condition & 20 (one per seed) \\
Statistical tests & Paired $t$-test \\
Multiple comparison correction & Bonferroni \\
Significance threshold & $\alpha = 0.05$ \\
\bottomrule
\end{tabular}
\end{table}

\subsection{Model Access}

All models were accessed via official APIs during [DATE RANGE]. We report exact model version strings in Table~\ref{tab:versions}. Note that API-served models may be updated by providers; we cannot guarantee identical behavior for future runs.

\subsection{Evaluation Protocol}

To reproduce our main results:
\begin{enumerate}
    \item Clone the repository and install dependencies: \texttt{uv pip install -e .}
    \item Set API keys: \texttt{export ANTHROPIC\_API\_KEY=...} (and similar for OpenAI, Google)
    \item Run evaluation: \texttt{lemonade batch configs/main\_experiment.yaml}
    \item Generate figures: \texttt{jupyter notebook analysis/generate\_figures.ipynb}
\end{enumerate}

Expected runtime for full replication: approximately [X] hours and \$[XXX] in API costs.



\end{document}

