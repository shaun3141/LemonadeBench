\section{Conclusion}
\label{sec:conclusion}

We introduced LemonadeBench, a benchmark for evaluating LLM agents on multi-day sequential business decision-making. Our key contributions include:

\begin{enumerate}
    \item A novel, interpretable environment requiring sustained strategic reasoning across 14 days with compounding decisions, perishable inventory, and stochastic weather.
    
    \item The first systematic study of goal-framing effects on LLM economic behavior, testing six conditions inspired by behavioral economics.
    
    \item Empirical evaluation of six frontier models revealing [key finding about model performance].
    
    \item Evidence that [key finding about framing effects], with implications for agent deployment and prompt engineering.
\end{enumerate}

Our results suggest that while current LLMs can engage in business reasoning, they [struggle with / show promise in] maintaining coherent long-horizon strategies. The sensitivity of agent behavior to goal framing raises important questions about the reliability of LLM agents in high-stakes decision contexts---and opportunities for steering agent behavior through careful prompt design.

LemonadeBench is publicly available at \url{https://github.com/Shaun3141/LemonadeBench}, including the environment, evaluation harness, all experimental data, and an interactive web client for human comparison.

\section*{Broader Impact Statement}

This work studies how LLM agents make economic decisions and how prompt framing influences their behavior. We identify both positive impacts and potential risks.

\paragraph{Positive Impacts.}
\begin{itemize}
    \item \textbf{Safer agent deployment}: Understanding agent economic biases \emph{before} deployment in real-world systems (e.g., automated trading, supply chain management) can prevent costly failures.
    \item \textbf{Alignment research}: Our findings on goal-framing effects contribute to understanding how LLMs interpret and act on objectives---a core challenge in AI alignment.
    \item \textbf{Educational value}: The interpretable lemonade stand domain provides an accessible testbed for teaching agent evaluation and behavioral economics concepts.
\end{itemize}

\paragraph{Potential Risks.}
\begin{itemize}
    \item \textbf{Manipulation of agent systems}: Knowledge of how goal framing affects agent behavior could be exploited to manipulate LLM-based systems (e.g., crafting prompts that induce excessive risk-taking in trading agents).
    \item \textbf{Overconfidence in benchmarks}: Strong performance on LemonadeBench does not guarantee safe behavior in more complex real-world economic environments.
    \item \textbf{Anthropomorphization}: Describing LLM behavior in terms of ``risk aversion'' or ``competitive drive'' may incorrectly suggest human-like cognition.
\end{itemize}

\paragraph{Mitigations.}
Our benchmark is explicitly designed as a low-stakes simulation with no real-world financial consequences. We emphasize throughout the paper that findings may not transfer to production systems without further validation. We release all code and data to enable scrutiny and reproducibility.

\section*{Acknowledgments}

% TODO: Acknowledgments

