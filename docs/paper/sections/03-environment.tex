\section{The LemonadeBench Environment}
\label{sec:environment}

LemonadeBench simulates a lemonade stand business over a configurable season (default: 14 days). The agent operates as a business owner making daily decisions about pricing, inventory procurement, location selection, and capital allocation. The environment follows the OpenEnv/Gymnasium API with \texttt{reset()} and \texttt{step(action)} methods, returning structured observations after each action.

The simulation models six interacting systems: (1) stochastic weather affecting customer demand, (2) price-sensitive customer behavior, (3) perishable inventory with expiration, (4) a reputation system with delayed feedback, (5) multiple locations with distinct economic properties, and (6) purchasable upgrades that modify game dynamics.

\subsection{State Space}
\label{sec:state}

Each observation $o_t$ contains approximately 35 fields organized into logical groups. Table~\ref{tab:observation} summarizes the key components.

\begin{table}[h]
\centering
\caption{LemonadeBench Observation Space}
\label{tab:observation}
\begin{tabular}{@{}lll@{}}
\toprule
\textbf{Category} & \textbf{Fields} & \textbf{Description} \\
\midrule
\textbf{Temporal} & \texttt{day}, \texttt{days\_remaining} & Current day (1--14) and remaining days \\
\textbf{Weather} & \texttt{weather}, \texttt{temperature} & Current conditions (5 types, 50--105°F) \\
 & \texttt{weather\_forecast} & Tomorrow's predicted weather \\
\midrule
\textbf{Financial} & \texttt{cash} & Current cash balance (cents) \\
 & \texttt{daily\_revenue}, \texttt{daily\_costs} & Today's income and expenses \\
 & \texttt{daily\_profit}, \texttt{total\_profit} & Daily and cumulative profit \\
\midrule
\textbf{Sales} & \texttt{cups\_sold}, \texttt{customers\_served} & Successful transactions \\
 & \texttt{customers\_turned\_away} & Lost sales due to stockouts \\
\midrule
\textbf{Inventory} & \texttt{lemons}, \texttt{sugar\_bags} & Current stock levels \\
 & \texttt{cups\_available}, \texttt{ice\_bags} & Consumable supplies \\
 & \texttt{lemons\_expiring\_tomorrow} & Perishables at risk \\
 & \texttt{lemons\_spoiled}, \texttt{ice\_melted} & Yesterday's losses \\
\midrule
\textbf{Reputation} & \texttt{reputation} & Customer perception (0.0--1.0) \\
 & \texttt{customer\_satisfaction} & Today's satisfaction score \\
\midrule
\textbf{Location} & \texttt{current\_location} & Active location (4 options) \\
 & \texttt{location\_catalog} & All locations with properties \\
\midrule
\textbf{Upgrades} & \texttt{owned\_upgrades} & Purchased improvements \\
 & \texttt{upgrade\_catalog} & Available purchases \\
\midrule
\textbf{Intelligence} & \texttt{market\_hints} & Demand forecasts, price curves \\
\bottomrule
\end{tabular}
\end{table}

The \texttt{market\_hints} field provides agents with actionable intelligence: estimated foot traffic ranges, price-to-conversion curves, revenue projections at each price point, and the current limiting resource for production. This transparency allows evaluation of strategic reasoning rather than reward hacking.

\subsection{Action Space}
\label{sec:actions}

Each action $a_t$ consists of eight parameters controlling daily operations:

\begin{table}[h]
\centering
\caption{LemonadeBench Action Space}
\label{tab:action}
\begin{tabular}{@{}llll@{}}
\toprule
\textbf{Parameter} & \textbf{Type} & \textbf{Range} & \textbf{Description} \\
\midrule
\texttt{price\_per\_cup} & int & 1--500 & Price in cents (e.g., 75 = \$0.75) \\
\texttt{buy\_lemons} & int & $\geq 0$ & Lemons to purchase (bulk discounts apply) \\
\texttt{buy\_sugar} & int & $\geq 0$ & Sugar bags to purchase \\
\texttt{buy\_cups} & int & $\geq 0$ & Disposable cups to purchase \\
\texttt{buy\_ice} & int & $\geq 0$ & Ice bags to purchase (melts overnight) \\
\texttt{advertising\_spend} & int & $\geq 0$ & Marketing budget in cents \\
\texttt{buy\_upgrade} & str & \{cooler, null\} & One-time equipment purchase \\
\texttt{location} & str & \{park, downtown, & Location for the day \\
 & & mall, pool, null\} & (null = stay at current) \\
\bottomrule
\end{tabular}
\end{table}

Only \texttt{price\_per\_cup} is required; all other parameters default to zero or null. Lemonade is produced \emph{on-demand} as customers arrive---agents do not pre-commit to production quantities, simplifying the decision space while maintaining inventory management complexity.

\paragraph{Action Validation.}
Actions are validated before execution. Invalid actions (e.g., purchases exceeding available cash) return an error response instead of advancing the day, allowing the agent up to 3 retry attempts per day. This design choice provides explicit feedback rather than silent failure, tests whether agents can correct mistakes given clear error messages, and enables measurement of ``constraint understanding''---a key reasoning capability. Error attempts are logged but do not consume game days, ensuring fair comparison across agents with different error rates.

\subsection{Environment Dynamics}
\label{sec:dynamics}

\paragraph{Weather System.}
Weather is sampled stochastically each day from five conditions with fixed probabilities: Hot (20\%), Sunny (35\%), Cloudy (25\%), Rainy (15\%), and Stormy (5\%). Each condition has an associated temperature range and demand multiplier (Table~\ref{tab:weather}). Critically, all randomness is pre-generated at episode start using a seed, ensuring identical weather sequences across runs with the same seed.

\begin{table}[h]
\centering
\caption{Weather Conditions and Effects}
\label{tab:weather}
\begin{tabular}{@{}lccl@{}}
\toprule
\textbf{Weather} & \textbf{Temp (°F)} & \textbf{Demand Mult.} & \textbf{Effect} \\
\midrule
Hot & 90--105 & 1.8$\times$ & Peak demand, ice bonus active \\
Sunny & 75--90 & 1.3$\times$ & Above-average demand \\
Cloudy & 65--80 & 0.9$\times$ & Slightly below average \\
Rainy & 55--70 & 0.4$\times$ & Low foot traffic \\
Stormy & 50--65 & 0.1$\times$ & Near-zero customers \\
\bottomrule
\end{tabular}
\end{table}

\paragraph{Customer Demand Model.}
Demand follows a two-stage funnel:

\begin{equation}
\text{Demand} = \underbrace{\text{FootTraffic}(w, \ell, r, a)}_{\text{visitors}} \times \underbrace{\text{Conversion}(p, w, i)}_{\text{purchase rate}}
\end{equation}

\noindent where $w$ is weather, $\ell$ is location, $r$ is reputation, $a$ is advertising spend, $p$ is price, and $i$ indicates ice availability.

Foot traffic is computed as:
\begin{equation}
\text{FootTraffic} = \text{BaseCustomers} \times \text{LocationMult} \times \text{WeatherMult} \times (0.5 + r) \times \text{AdBonus} \times \epsilon
\end{equation}
\noindent where $\epsilon \sim \mathcal{U}(0.9, 1.1)$ adds daily variance (pre-computed per seed).

Conversion rate decreases with price above the optimal point (\$0.50):
\begin{equation}
\text{Conversion} = 0.95 \times \max\left(0.1, 1 - \left(\frac{p - 50}{100}\right)^{0.7} \times s \times 50\right)
\end{equation}
\noindent where $s$ is the location's price sensitivity. On hot days, ice provides a 20\% conversion bonus.

\paragraph{Perishable Inventory.}
Lemons expire after 3 days using FIFO consumption. Ice melts completely overnight unless the agent owns a Cooler upgrade (which preserves 50\%). This creates pressure to forecast demand accurately---overbuying results in spoilage losses.

\paragraph{Reputation System.}
Reputation $r_t$ evolves as an exponential moving average:
\begin{equation}
r_{t+1} = 0.8 \cdot r_t + 0.2 \cdot s_t
\end{equation}
\noindent where $s_t$ is today's customer satisfaction (based on pricing and stockout rate). This delayed feedback rewards consistent performance over short-term exploitation.

\paragraph{Locations.}
Four locations offer strategic tradeoffs (Table~\ref{tab:locations}). Switching locations incurs a one-time permit fee.

\begin{table}[h]
\centering
\caption{Location Properties}
\label{tab:locations}
\begin{tabular}{@{}lcccc@{}}
\toprule
\textbf{Location} & \textbf{Traffic} & \textbf{Price Sens.} & \textbf{Weather Exp.} & \textbf{Permit} \\
\midrule
Park & 1.2$\times$ & 0.018 & 1.0$\times$ & Free \\
Downtown & 1.0$\times$ & 0.012 & 0.7$\times$ & \$10.00 \\
Mall & 0.7$\times$ & 0.008 & 0.0$\times$ & \$15.00 \\
Pool & 0.9$\times$ & 0.020/0.010* & 1.8$\times$ & \$2.50 \\
\bottomrule
\end{tabular}
\begin{flushleft}
\small *Pool uses reduced price sensitivity (0.010) on hot/sunny days.
\end{flushleft}
\end{table}

\paragraph{Reward Structure.}
The reward at each step is daily profit divided by 100 (converting cents to dollars). At episode end, a bonus of $\text{total\_profit} / 1000$ encourages long-term optimization over myopic daily gains.
